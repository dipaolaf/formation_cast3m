%%%%%%%%%%%%%%%%%%%%%%%%%%%%%%%%%%%%%%%%%%%%%%%%%%%%%%%%%%%%%%%%%%%%%
\fe{\section{La procédure PASAPAS}}{\section{The PASAPAS procedure}}
\label{pasapas}
%%%%%%%%%%%%%%%%%%%%%%%%%%%%%%%%%%%%%%%%%%%%%%%%%%%%%%%%%%%%%%%%%%%%%

\begin{frame}{\fe{Algorithme de PASAPAS}{PASAPAS algorithm}}
  \begin{itemize}
    \item \fe{Initialisations}{Initializations}
    \item \fe{\green{Boucle sur les pas de temps}}{\green{Loop on time steps}}
    \begin{itemize}
      \item \fe{\red{Boucle de convergence thermo mécanique}}{\red{Loop for themo mechanical convergence}}
      \begin{itemize}
        \item \fe{\red{Solveur thermique}}{\red{Thermal solver}}
        \item \fe{\red{Solveur thermique}}{\red{Thermal solver}}
      \end{itemize}
      \item \fe{\red{Convergence thermo mécanique ?}}{\red{Thermo mechanical convergence?}}
    \end{itemize}
    \item \fe{\green{Enregistrement des résultats}}{\green{Save results}}
    \item \fe{Fin}{End}
    \footnotesize
   \end{itemize}
  \begin{center}
    \fe{$\Rightarrow$ Sous procédures pour certaines tâches}{$\Rightarrow$ Sub pocedures for some tasks}\\
    \kwr{PAS\_DEFA~~PAS\_INIT~~PAS\_ETAT~~PAS\_MATE~~PAS\_EPTH}\\
    \kwr{PAS\_MODL~~TRANSNON~~PAS\_RAYO~~UNPAS~~PAS\_RESU}\\~\\
    \fe{$\Rightarrow$ Appels optionnels à des procédures utilisateur}{$\Rightarrow$ Optional user procedures calls}\\
    \kwv{PERSO1~~~PERSO2~~~REEV\_MEC~~~REEV\_THE}\\
    \kwv{CHARMECA~~~CHARTHER~~~PARATHER}\\
    \fe{ces procédures sont vides et à définir par l'utilisateur !}{these procedures must be defined by the user!}
  \end{center}
  \normalsize
\end{frame}

\begin{frame}{\fe{Algorithme de PASAPAS}{PASAPAS algorithm}}
  \tiny
  \begin{tikzpicture}[node distance=0.6cm,
    every node/.style={fill=white}, align=center]
  % Cellules (style, position, contenu)
  \fe{\node (start)  [bstep] {Initialisations,\\ \kwv{PERSO1}, \kwv{REEV\_MEC}, \kwv{REEV\_THE}};}
     {\node (start)  [bstep] {Initializations,\\ \kwv{PERSO1}, \kwv{REEV\_MEC}, \kwv{REEV\_THE}};}
  \fe{\node (start0) [note, right of=start, xshift=5cm] {\kwr{PAS\_DEFA} (lecture des options),\\
                                                         \kwr{PAS\_INIT PAS\_ETAT PAS\_MATE} (initialisations matériaux, inconnues,\\
                                                         chargements, variables temporaires), remplissage de \kw{tab1.}\kwg{'WTABLE'}};}
     {\node (start0) [note, right of=start, xshift=5cm] {\kwr{PAS\_DEFA} (reading options),\\
                                                         \kwr{PAS\_INIT PAS\_DEFA PAS\_MATE} (initializations materials, unknowns,\\
                                                         loading, temporary variables), filling \kw{tab1.}\kwg{'WTABLE'}};}

  \fe{\node (init)  [gstep, below of=start, yshift=-0.1cm] {Initialisation du pas de temps};}
     {\node (init)  [gstep, below of=start, yshift=-0.1cm] {Time step initialization};}
  \fe{\node (init0) [note, right of=init, xshift=5.28cm] {\kwr{PAS\_MODL PAS\_ETAT PAS\_MATE} Mise à jour modèle, matériau, chargement};}
     {\node (init0) [note, right of=init, xshift=4.9cm] {\kwr{PAS\_MODL PAS\_ETAT PAS\_MATE} Updating model, properties, load};}

  \fe{\node (resot) [rstep, below of=init] {\g{Résolution Thermique}};}
     {\node (resot) [rstep, below of=init] {\g{Thermal solving}};}
  \fe{\node (resot0) [note, right of=resot, xshift=5.1cm] {\kwr{TRANSNON} Solveur thermique sur le pas de temps\\
                                                           (aussi dispo : \kwr{TRANSLIN}, \kwr{DUPONT2})\\
                                                           Écriture des résultats dans \kw{tab1.}\kwg{'ESTIMATION'}};}
     {\node (resot0) [note, right of=resot, xshift=4.9cm] {\kwr{TRANSNON} Thermal solver on the time step\\
                                                           (also available: \kwr{TRANSLIN}, \kwr{DUPONT2})\\
                                                           Saving results in \kw{tab1.}\kwg{'ESTIMATION'}};}

  \node (rthe)  [rstep, below of=resot] {\kwv{REEV\_THE}};
  \fe{\node (rthe0) [note, right of=rthe, xshift=4.94cm] {Appel à la procédure \kwv{REEV\_THE} (si demandé)};}
     {\node (rthe0) [note, right of=rthe, xshift=4.87cm] {Call to the \kwv{REEV\_THE} procédure (if asked)};}

  \fe{\node (resom) [rstep, below of=rthe]  {\g{Résolution Mécanique}};}
     {\node (resom) [rstep, below of=rthe]  {\g{Mechanical solving}};}
  \fe{\node (resom0) [note, right of=resom, xshift=5.08cm] {\kwr{UNPAS} Solveur mécanique sur le pas de temps\\
                                                           Écriture des résultats dans \kw{tab1.}\kwg{'ESTIMATION'}};}
     {\node (resom0) [note, right of=resom, xshift=4.87cm] {\kwr{UNPAS} Mechanical solver on the time step\\
                                                           Saving results in \kw{tab1.}\kwg{'ESTIMATION'}};}

  \node (rmec)  [rstep, below of=resom] {\kwv{REEV\_MEC}};
  \fe{\node (rmec0) [note, right of=rmec, xshift=4.94cm] {Appel à la procédure \kwv{REEV\_MEC} (si demandé)};}
     {\node (rmec0) [note, right of=rmec, xshift=4.82cm] {Call to the \kwv{REEV\_MEC} procedure (if asked)};}

  \node (contm) [rtest, below of=rmec, yshift=-0.3cm]  {Convergence ?};
  \fe{\node (contm0) [note, right of=contm, xshift=5.5cm] {Test de stationnarité, à activer avec :\\
                                                           \kw{tab1.}\kwg{'CONVERGENCE\_MEC\_THE'}\kw{ = VRAI ;}\\ \\
                                                           $\frac{\max \|T_i-T_{i-1}\|}{\max \|T_i\|} \le$ \kw{tab1.}\kwg{'CRITERE\_COHERENCE'} $(10^{-2})$};}
     {\node (contm0) [note, right of=contm, xshift=5.61cm] {Stationarity test, to be activated with:\\
                                                           \kw{tab1.}\kwg{'CONVERGENCE\_MEC\_THE'}\kw{ = VRAI ;}\\ \\
                                                           $\frac{\max \|T_i-T_{i-1}\|}{\max \|T_i\|} \le$ \kw{tab1.}\kwg{'CRITERE\_COHERENCE'} $(10^{-2})$};}

  \fe{\node (prep)  [gstep, below of=contm, yshift=-0.4cm] {Préparation du pas de temps suivant};}
     {\node (prep)  [gstep, below of=contm, yshift=-0.4cm] {Next time step preparation};}
  \fe{\node (prep0) [note, right of=prep, xshift=5.15cm] {Enregistrement du nouvel état au début de pas dans : \kw{tab1.}\kwg{'WTABLE'}};}
     {\node (prep0) [note, right of=prep, xshift=5.15cm] {Saving the new state as the initial time step state in: \kw{tab1.}\kwg{'WTABLE'}};}

  \node (pers)  [gstep, below of=prep]  {\kwv{PERSO1}};
  \fe{\node (pers0) [note, right of=pers, xshift=3.8cm] {Appel à la procédure \kwv{PERSO1} (si demandé)};}
     {\node (pers0) [note, right of=pers, xshift=3.7cm]    {Call to the \kwv{PERSO1} procedure (if asked)};}

  \fe{\node (sauv)  [gstep, below of=pers] {Sauvegarde des résultats};}
     {\node (sauv)  [gstep, below of=pers] {Saving results};}
  \fe{\node (sauv0) [note, right of=sauv, xshift=4.93cm] {\kwr{PAS\_RESU} : remplissage de \kwg{'CONTINUATION'} d'après \kwg{'ESTIMATION'}\\
                                                         écriture dans \kw{tab1} s'il s'agit d'un \kwg{'TEMPS\_SAUVE'}};}
     {\node (sauv0) [note, right of=sauv, xshift=4.8cm] {\kwr{PAS\_RESU}: filling \kwg{'CONTINUATION'} according to \kwg{'ESTIMATION'}\\
                                                         saving in \kw{tab1} if the time belong to \kwg{'TEMPS\_SAUVE'}};}

  \fe{\node (cont)  [gtest, below of=sauv, , yshift=-0.3cm]  {Fin des pas\\ de temps ?};}
     {\node (cont)  [gtest, below of=sauv, , yshift=-0.3cm]  {End of\\ time steps?};}
  \fe{\node (end)   [bstep, right of=cont, , xshift=2cm]  {Fin};}
     {\node (end)   [bstep, right of=cont, , xshift=2cm]  {End};}
% Connections entre les cellules
  \draw[->]             (start) -- (init);
  \draw[->, draw=Green] (init) -- (resot);
  \draw[->, draw=red]   (resot) -- (rthe);
  \draw[->, draw=red]   (rthe) -- (resom);
  \draw[->, draw=red]   (resom) -- (rmec);
  \draw[->, draw=red]   (rmec) -- (contm);
  \fe{\draw[->, draw=Green] (contm) -- node[xshift=0.4cm, yshift=0.1cm,]{oui} (prep);}
     {\draw[->, draw=Green] (contm) -- node[xshift=0.4cm, yshift=0.1cm,]{yes} (prep);}
  \fe{\draw[->, draw=red]   (contm.west) -- node[yshift=0.2cm]{non} ++(-0.55,0) -- ++(0,2.7) -- (resot.west);}
     {\draw[->, draw=red]   (contm.west) -- node[yshift=0.2cm]{no}  ++(-0.55,0) -- ++(0,2.7) -- (resot.west);}
  \draw[->, draw=Green] (prep) -- (pers);
  \draw[->, draw=Green] (pers) -- (sauv);
  \draw[->, draw=Green] (sauv) -- (cont);
  \fe{\draw[->]             (cont) -- node[yshift=0.2cm,]{oui} (end);}
     {\draw[->]             (cont) -- node[yshift=0.2cm,]{yes} (end);}
  \fe{\draw[->, draw=Green] (cont.west) -- node[yshift=0.2cm]{non} ++(-0.6,0) -- ++(0,6.4) -- (init.west);}
     {\draw[->, draw=Green] (cont.west) -- node[yshift=0.2cm]{no} ++(-0.6,0) -- ++(0,6.4) -- (init.west);}
  \draw[dotted]         (start) -- (start0);
  \draw[dotted]         (init)  -- (init0);
  \draw[dotted]         (resot) -- (resot0);
  \draw[dotted]         (rthe)  -- (rthe0);
  \draw[dotted]         (resom) -- (resom0);
  \draw[dotted]         (rmec)  -- (rmec0);
  \draw[dotted]         (contm) -- (contm0);
  \draw[dotted]         (prep)  -- (prep0);
  \draw[dotted]         (pers)  -- (pers0);
  \draw[dotted]         (sauv)  -- (sauv0);
  \end{tikzpicture}
\end{frame}

\begin{frame}{\fe{Accès aux données de PASAPAS}{Acces to PASAPAS data}}
  \begin{itemize}
    \item \kw{tab1.}\kwg{'ESTIMATION'}\\
    \footnotesize
    \fe{contient les résultats calculés par \kwr{TRANSNON} et \kwr{UNPAS}\\
        mais non convergées dans la \red{boucle de stationnarité thermo mécanique}}
       {contains all results calculated by \kwr{TRANSNON} and \kwr{UNPAS}\\
        but not converged for the \red{thermo mechanical stationarity loop}}
    \lstinputlisting[language=gibiane, firstline=18, lastline=22]{dgibi/exemples.dgibi}
    \normalsize
    \item \kw{tab1.}\kwg{'CONTINUATION'}\\
    \footnotesize
    \fe{contient les résultats calculés et convergés (pour la \red{boucle thermo mécanique})\\
        cet indice est mis à jour à la fin du pas de temps !\\
        utile pour une reprise de \kwr{PASAPAS}}
       {contains converged results for the \red{thermo mechanical loop}\\
        updated at the end of the time step!\\
        useful for \kwr{PASAPAS} restart}
    \lstinputlisting[language=gibiane, firstline=23, lastline=27]{dgibi/exemples.dgibi}
    \normalsize
  \end{itemize}
\end{frame}

\begin{frame}{\fe{Accès aux données de PASAPAS}{Acces to PASAPAS data}}
  \begin{itemize}
    \item \kw{tab1.}\kwg{'WTABLE'}\\
    \footnotesize
    \fe{Variables utiles à \kwr{PASAPAS} (options choisies, modèles, caractéristiques instanciées,\\
        chargements instanciés, résultats intermédiaires, \dots)}
       {All useful variables for \kwr{PASAPAS} (chosen options, models, INSTANCIEES properties,\\
        UPDATED loads, intermediate results, \dots)}\\~\\
    \fe{Quelques indices :}{Some indexes}
    \scriptsize
    \begin{tabular}{ll}
      \kwg{'WTABLE'} \kw{.} \kwg{'CHARGEMENT'}           & \fe{Chargement}{Loading}\\
      \kwg{'WTABLE'} \kw{.} \kwg{'THER\_COURANT'}        & \fe{Température à la dernière itération}{Value of temperature at last iteration}\\
                                                         & \fe{(au cours d'un pas)}{(during a time step)}\\
      \kwg{'WTABLE'} \kw{.} \kwg{'BLOCAGES\_MECANIQUES'} & \fe{Matrice de blocage mécanique}{Mechanical constraint matrix}\\
      \kwg{'WTABLE'} \kw{.} \kwg{'BLOCAGES\_THERMIQUES'} & \fe{Matrice de blocage thermique}{Thermal constraint matrix}\\
      \kwg{'WTABLE'} \kw{.} \kwg{'FOR'}                  & \fe{Configuration au début du pas}{Configuration at the time step beginning}\\
      \kwg{'WTABLE'} \kw{.} \kwg{'FOR0'}                 & \fe{Configuration initiale}{Initial configuration}\\
      \kwg{'WTABLE'} \kw{.} \kwg{'MODELE'}               & \fe{Modèles}{Modeles}\\
      \kwg{'WTABLE'} \kw{.} \kwg{'CARACTERISTIQUES'}     & \fe{Champ de caractéristiques matériau}{Material properties field}\\
      etc\dots&\\
    \end{tabular}
    \\~\\ \fe{Plus d'infos, voir les commentaires procédure \kwr{PAS\_DEFA}}
             {More information in the comments of the procedure \kwr{PAS\_DEFA}}
    \normalsize
  \end{itemize}
\end{frame}

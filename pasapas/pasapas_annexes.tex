%%%%%%%%%%%%%%%%%
\section{Annexes}
\label{annexe}
%%%%%%%%%%%%%%%%%

\begin{frame}{\fe{Annexe : Paramètres de contrôle}{Annex: Control parameters}}
  \begin{itemize}
    \item \fe{Généralités}{General}
  \end{itemize}
  \tiny
  \hspace{0.4cm}
  \begin{tabular}{lll}
    \kwg{NB\_BOTH}     & ENTIER & \fe{Nombre max. d'itérations de la boucle de convergence thermo mécanique (\kw{10})}
                                     {Maximum number of iterations for the thermo mechanical convergence loop (\kw{10})}\\
    \kwg{MAXITERATION} & ENTIER & \fe{Nombre max. d'itérations (\kw{49})}
                                     {Maximum number of iterations (\kw{49})}
  \end{tabular}
  \normalsize
  \begin{itemize}
    \item \fe{Mécanique}{Mechanics}\\
  \end{itemize}
  \tiny
  \hspace{0.4cm}
  \begin{tabular}{lll}
    \kwg{PRECISION}            & FLOTTANT & \fe{Critère pour comparer le résidu (\kw{1.E-4})}
                                               {Criterion to compare the imbalance (\kw{1.E-4})}\\
    \kwg{FTOL}                 & FLOTTANT & \fe{Tolérance pour l'équilibre des efforts}
                                               {Tolerance for strength equilibrium}\\
    \kwg{MTOL}                 & FLOTTANT & \fe{Tolérance pour l'équilibre des moments}
                                               {Tolerance for moments equilibrium}\\
    \kwg{GRANDS\_DEPLACEMENTS} & LOGIQUE  & \fe{Configuration de référence = configuration déformée (\kw{FAUX})}
                                               {Reference configuration = deformed configuration (\kw{FAUX})}\\
    \kwg{PREDICTEUR}           & MOT      & \fe{\kw{= }\kwg{HPP}}{\kw{= }\kwg{HPP}}\\
                               &          & \fe{Calcul en hypothèse "petits" déplacements $\rightarrow$ 1ère convergence}
                                               {Using "small" displacements hypothesis $\rightarrow$ 1st convergence}\\
                               &          & \fe{puis passage grands déplacements $\rightarrow$ 2ème convergence}
                                               {then use the large displacements hypothesis $\rightarrow$ 2nd convergence}\\
    \kwg{PRECISINTER}          & FLOTTANT & \fe{Précision pour le problème d'intégration des lois constitutives (\kw{1.E-8})}
                                               {Precision for the constitutive laws integration problem (\kw{1.E-8})}\\
    \kwg{CONVERGENCE\_FORCEE}  & LOGIQUE  & \fe{Utilisation, ou non, de la convergence forcée (\kw{VRAI})}
                                               {Forced convergence use, or not (\kw{VRAI})}\\
    \kwg{MAXSOUSPAS}           & ENTIER   & \fe{Nombre max. de sous pas en convergence forcée (\kw{100})}
                                               {Max. number of sub steps during forced convergence (\kw{100})}\\
    \kwg{DELTAITER}            & ENTIER   & \fe{Nombre de pas d'écart pour test de non convergence (\kw{10})}
                                               {Number of steps over which non convergence is tested (\kw{10})}
  \end{tabular}
\end{frame}

\begin{frame}{\fe{Annexe : Paramètres de contrôle}{Annex: Control parameters}}
  \begin{itemize}
    \item \fe{Thermique}{Thermal}
  \end{itemize}
  \tiny
  \hspace{0.4cm}
  \begin{tabular}{lll}
    \kwg{PROCEDURE\_THERMIQUE} & MOT      & \fe{Procédure de calcul à utiliser :}
                                               {Procedure to call:}\\
                               &          & \fe{\kw{= }\kwg{NONLINEAIRE} procédure \kwo{TRANSNON} (défaut)}
                                               {\kw{= }\kwg{NONLINEAIRE} procedure \kwo{TRANSNON} (default)}\\
                               &          & \fe{\kw{= }\kwg{LINEAIRE} procédure \kwo{TRANSLIN}}
                                               {\kw{= }\kwg{LINEAIRE} procedure \kwo{TRANSLIN}}\\
                               &          & \fe{\kw{= }\kwg{DUPONT} procédure \kwo{DUPONT2}}
                                               {\kw{= }\kwg{DUPONT} procedure \kwo{DUPONT2}}\\
    \kwg{RELAXATION\_THETA}    & FLOTTANT & \fe{Coefficient de relaxation pour la $\theta$-méthode (\kw{1})}
                                               {Relaxation coefficient for the $\theta$-method (\kw{1})}\\
  \end{tabular}
  \normalsize
  \begin{itemize}
    \item \fe{Couplage thermo mécanique}{Thermo mechanical coupling}\\
  \end{itemize}
  \tiny
  \hspace{0.4cm}
  \begin{tabular}{lll}
    \kwg{CONVERGENCE\_MEC\_THE} & LOGIQUE  & \fe{Indique que l'on souhaite ré itérer la boucle de calcul thermo mécanique}
                                                {Indicates that the thermo mechanical loop should be repeated}\\
                                &          & \fe{en cas de dépendance mutuelle (\kw{FAUX})}
                                                {in case of dependence (\kw{FAUX})}\\
    \kwg{CRITERE\_COHERENCE}    & FLOTTANT & \fe{Précision pour la convergence thermo mécanique,}
                                                {Precision for the thermo mechanical convergence,}\\
                                &          & \fe{test sur la thermique (\kw{= }\kwg{PRECISION})}
                                                {tested on thermal results (\kw{= }\kwg{PRECISION})}\\
    \kwg{PROJECTION}            & LOGIQUE  & \fe{Indique que le problème est couplé mais que les maillages en mécanique}
                                                {Indicates that the thermal and mechanical meshes are different}\\
                                &          & \fe{et en thermique sont différents (\kw{FAUX})}
                                                {while the problem is coupled (\kw{FAUX})}
  \end{tabular}
\end{frame}

\begin{frame}{\fe{Annexe : Minimisation du résidu en Gibiane}{Annex: Imbalance minimization in Gibiane}}
  \begin{textblock*}{8cm}(5.7cm,0cm)
    \lstinputlisting[basicstyle=\ttfamily\tiny,language=gibiane, firstline=47, lastline=73]{dgibi/exemples.dgibi}
  \end{textblock*}
  \begin{itemize}
    \footnotesize
    \item \fe{Algorithme \kwo{UNPAS} simplifié\\
              $\rightarrow$ voir la procédure \kwo{@SOLVMEC}\\~}
             {Simplified \kwo{UNPAS} algorithm\\
              $\rightarrow$ see the \kwo{@SOLVMEC} procedure\\~}
  \end{itemize}
  \vspace{6cm}
\end{frame}

\begin{frame}{\fe{Annexe : Minimisation du résidu en Gibiane}{Annex: Imbalance minimization in Gibiane}}
  \begin{textblock*}{8cm}(5.7cm,0cm)
    \lstinputlisting[basicstyle=\ttfamily\tiny,language=gibiane, firstline=80, lastline=106]{dgibi/exemples.dgibi}
  \end{textblock*}
  \begin{itemize}
    \footnotesize
    \item \fe{Algorithme \kwo{UNPAS} simplifié\\
              $\rightarrow$ voir la procédure \kwo{@SOLVMEC}\\
              grand déplacements}
             {Simplified \kwo{UNPAS} algorithm\\
              $\rightarrow$ see the \kwo{@SOLVMEC} procedure\\
              large displacements}
  \end{itemize}
  \vspace{6cm}
\end{frame}

\begin{frame}{\fe{Annexe : Critères de convergence (UNPAS)}{Annex: Convergence criteria (UNPAS)}}
  \begin{itemize}
    \item \fe{Normes de convergence (après le 1er \kwr{RESO})}{Convergence norms (after the 1st \kwr{RESO})}
  \end{itemize}
  \tiny
  \begin{align*}
    F_{\tx{ref}} = & \frac{\bigg|\violet{\delta\{U\}_1^1}.\{F\}_{\tx{imp}}-\red{\{\lambda\}_1^1}.\left(\{u\}_{\tx{imp}}-[A]^T.\red{\{U\}_1^0}\right)\bigg|}
                          {\underset{\tx{ddl depl}}{\max} \Big|\violet{\delta\{U\}_1^1}\Big|+\kw{xpetit}}
                   + \frac{\underset{\tx{ddl depl}}{\max} \Big|\violet{\delta\{U\}_1^1}\Big|.\tx{max}\Big|[B]\red{\{\sigma\}_1^0}\Big|}
                          {\underset{\tx{ddl depl}}{\max} \Big|\violet{\delta\{U\}_1^1}\Big|+\kw{xpetit}}
                   + \max\Big|[B]\red{\{\sigma\}_1^0}\Big|\\
    M_{\tx{ref}} = & \frac{\bigg|\violet{\delta\{U\}_1^1}.\{F\}_{\tx{imp}}-\red{\{\lambda\}_1^1}.\left(\{u\}_{\tx{imp}}-[A]^T.\red{\{U\}_1^0}\right)\bigg|}
                          {\underset{\tx{ddl rota}}{\max} \Big|\violet{\delta\{U\}_1^1}\Big|+\kw{xpetit}}
                   + \frac{\underset{\tx{ddl depl}}{\max} \Big|\violet{\delta\{U\}_1^1}\Big|.\tx{max}\Big|[B]\red{\{\sigma\}_1^0}\Big|}
                          {\underset{\tx{ddl rota}}{\max} \Big|\violet{\delta\{U\}_1^1}\Big|+\kw{xpetit}}
                   + \kw{xpetit}
  \end{align*}
  \normalsize
  \begin{itemize}
    \item \fe{Mesures du résidu (à chaque itération de \kwo{UNPAS})}{Imbalace measure (at each iteration on \kwo{UNPAS})}
  \end{itemize}
  \tiny
  \begin{align*}
    \kw{xconv}  = & \frac{\max\biggl\{\underset{\tx{ddl depl}}{\max}\Big|\{F\}_{\tx{imp}}-[A]^T.\red{\{\lambda\}_1^{i+1}}-[B].\red{\{\sigma\}_1^{i+1}}\Big|~;~\max\bigg|[A]^T\left(\red{\{\lambda\}_1^{i+1}}-\red{\{\lambda\}_1^i}\right)\bigg|\biggl\}}
                        {F_{\tx{ref}}}\\
    \kw{xconvm} = & \frac{\underset{\tx{ddl rota}}{\max}\Big|\{F\}_{\tx{imp}}-[A]^T.\red{\{\lambda\}_1^{i+1}}-[B].\red{\{\sigma\}_1^{i+1}}\Big|}
                        {M_{\tx{ref}}}
  \end{align*}
\end{frame}

\begin{frame}{\fe{Annexe : Critères de convergence (UNPAS)}{Annex: Convergence criteria (UNPAS)}}
  \begin{itemize}
    \item \fe{Mesure de la variation d'incrément de déformation entre 2 itérations}
             {Measure of strain increment variation between 2 iterations}
  \end{itemize}
  \begin{equation*}
    \kw{depstdm} = \Big|\red{\Delta\{\varepsilon\}_1^{i+1}-\Delta\{\varepsilon\}_1^i}\Big|
  \end{equation*}
  \begin{itemize}
    \item \fe{La convergence est atteinte si :}{Convergence is reached if:}\\
    \center
    \begin{tabular}{rl}
      \kw{xconv}   & $<$~~\kwg{PRECISION}\\
      \kw{xconvm}  & $<$~~\kwg{PRECISION}\\
      \kw{depstdm} & $<$~~\kwg{PRECISION}
    \end{tabular}
  \end{itemize}
\end{frame}

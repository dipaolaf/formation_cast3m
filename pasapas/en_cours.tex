% Remplacer les apostrophes ’ par des quotes '
% Exo 1 : mettre les grands deplacements initialement
% Du gras dans les objectifs
% Certaines slides mordent la bas des pages
% Vérifier la version anglaise
% Valeurs par défauts des parametres annexe
% Tableaux des rappels a aligner comme ceux de l'annexe !


%%%%%%%%%%%%%%%%%
\section{Annexes}
\label{annexe}
%%%%%%%%%%%%%%%%%

\begin{frame}{\fe{Annexe : paramètres de contrôle}{Annex: Control parameters}}
  \begin{itemize}
    \item \fe{Généralités}{General}
  \end{itemize}
  \tiny
  \hspace{0.4cm}
  \begin{tabular}{lll}
    \kwg{'NB\_BOTH'}     & ENTIER & \fe{Nombre max. d'itérations de la boucle de convergence thermo mécanique}
                                       {Maximum number of iterations for the thermo mechanical convergence loop}\\
    \kwg{'MAXITERATION'} & ENTIER & \fe{Nombre max. d'itérations (\kw{49})}
                                       {Maximum number of iterations (\kw{49})}
  \end{tabular}
  \normalsize
  \begin{itemize}
    \item \fe{Mécanique}{Mechanics}\\
  \end{itemize}
  \tiny
  \hspace{0.4cm}
  \begin{tabular}{lll}
    \kwg{'PRECISION'}            & FLOTTANT & \fe{Critère pour comparer le résidu (\kw{1.E4})}
                                                 {Criterion to compare the imbalance (\kw{1.E4})}\\
    \kwg{'FTOL'}                 & FLOTTANT & \fe{Tolérance pour l’équilibre des efforts}
                                                 {Tolerance for strength equilibrium}\\
    \kwg{'MTOL'}                 & FLOTTANT & \fe{Tolérance pour l’équilibre des moments}
                                                 {Tolerance for moments equilibrium}\\
    \kwg{'GRANDS\_DEPLACEMENTS'} & LOGIQUE  & \fe{Configuration de référence = configuration déformée}
                                                 {Reference configuration = deformed configuration}\\
    \kwg{'PREDICTEUR'}       & \kw{= 'HPP'} & \fe{Calcul en hypothèse "petits" déplacements $\rightarrow$ 1ère convergence}
                                                 {Using "small" displacements hypothesis $\rightarrow$ 1st convergence}\\
                             &              & \fe{puis passage grands déplacements $\rightarrow$ 2ème convergence}
                                                 {then use the large displacements hypothesis $\rightarrow$ 2nd convergence}\\
    \kwg{'PRECISINTER'}          & FLOTTANT & \fe{Précision pour le problème d’intégration des lois constitutives (\kw{1.E8})}
                                                 {Precision for the constitutive laws integration problem (\kw{1.E8})}\\
    \kwg{'CONVERGENCE\_FORCEE'}  & LOGIQUE  & \fe{Utilisation, ou non, de la convergence forcée (\kw{VRAI})}
                                                 {Forced convergence use, or not (\kw{VRAI})}\\
    \kwg{'MAXSOUSPAS'}           & ENTIER   & \fe{Nombre max. de sous pas en convergence forcée (\kw{200})}
                                                 {Max. number of sub steps during forced convergence (\kw{200})}\\
    \kwg{'DELTAITER'}            & ENTIER   & \fe{Nombre de pas d'écart pour test de non convergence}
                                                 {Number of steps over which non convergence is tested}
  \end{tabular}
\end{frame}

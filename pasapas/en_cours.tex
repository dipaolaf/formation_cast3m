% Décomposer en \subsection la méca :
%   1) équations + formulation faible, minimisation du résidu
%   2) la procédure UNPAS
%   3) exemples

% Faire la thermique en premier, puis la mécanique ??

% Remplacer les apostrophes ’ par des quotes '

{\setbeamerfont{framesubtitle}{size=\tiny}
\begin{frame}{\fe{Exercice 1 : poutre avec force suiveuse}{Exercise 1: beam with following force}}
             {\url{https://www-cast3m.cea.fr/index.php?page=exemples&exemple=formation_pasapas_1_initial}}
  \small
  \begin{itemize}
    \item \fe{Poutre en flexion}{Beam bending}\\
    \scriptsize
    \fe{base encastrée, force \tou{perpendiculaire} à la poutre, déplacement "important"}
       {clamped base, force \tou{perpendicular} to the beam, "large" displacement}\\
    \begin{tikzpicture}
      \draw [-,thick] (0,0) -- (0,3);
      \draw [->, red] (0,3) -- (0.5,3) node (force) [right, red] {$F$};
      \draw [-]       (-0.2,0) -- (0.25,0);
      \draw [-]       (-0.2,-0.1) -- (-0.1,0);
      \draw [-]       (-0.1,-0.1) -- (0,0);
      \draw [-]       (0,-0.1) -- (0.1,0);
      \draw [-]       (0.1,-0.1) -- (0.2,0);
    \end{tikzpicture}
    \normalsize
    \item \fe{\red{Problème : l'effort est calculé sur la configuration initiale et n'est pas mis à jour}}
             {\red{Problem: the force is calculated on the initial shape and not updated}}
    \item \fe{\green{Objectif : réappliquer l'effort correctement au cours des pas}}
             {\green{Purpose: to correctly apply the force during the calculation}}
    \begin{center}
      \fe{\avous{~à vous de jouer !}}{\avous{~it's up to you!}}
    \end{center}  
  \end{itemize}
\end{frame}
}

%%%%%%%%%%%%%%%%%%%%%%%%%%%%%%%%%%%%%%%%%%%%%%%%%%%%%%%%%%%%%
\fe{\section{Solveur mécanique}}{\section{Mechanical solver}}
\label{mecanique}
%%%%%%%%%%%%%%%%%%%%%%%%%%%%%%%%%%%%%%%%%%%%%%%%%%%%%%%%%%%%%

\fe{\subsection{Équations}}{\subsection{Equations}}
\label{mecanique_equations}

\begin{frame}{\fe{Solveur mécanique}{Mechanical solver}}
             {\fe{Rappel des équations}{Reminder on equations}}
  \begin{itemize}
    \item<1->[] \fe{Équations locales du problème statique}
                   {Local equations of the static equilibrium}\\
    \scriptsize
    \fe{\begin{tabular}{lll}
          équilibre            & $\tx{div}(\tod{\sigma})+f_{\tx{imp}}=0$ & sur $\partial V$ \\
          efforts imposés      & $\tod{\sigma}.n=t_{\tx{imp}}$           & sur $\partial V_t$ \\
          déplacements imposés & $u=u_{\tx{imp}}$                        & sur $\partial V_u$ \\
        \end{tabular}}
       {\begin{tabular}{lll}
        equilibrium           & $\tx{div}(\tod{\sigma})+f_{\tx{imp}}=0$ & on $\partial V$ \\
        imposed forces        & $\tod{\sigma}.n=t_{\tx{imp}}$           & on $\partial V^t$ \\
        imposed displacements & $u=u_{\tx{imp}}$                        & on $\partial V^u$ \\
      \end{tabular}}
    \normalsize
    \onslide<2->{
    \begin{block}{\fe{Formulation faible + discrétisation EF}{Weak form + FE discretization}}
      \footnotesize
      \begin{equation*}
        \{F\}_{\tx{ext}}-\int_V[B]^T\{\sigma\}dV=\{0\}
      \end{equation*}
      \scriptsize
      \begin{equation*}
        \underbrace{\int_{\partial V_t}[N]^T\{t\}_{\tx{imp}}dS + \int_V[N]^T\{f\}_{\tx{imp}}dV}_{\{F\}_{\tx{imp}}} +
        \underbrace{\int_{\partial V_u}[N]^T\{\tod{\sigma}.n\}dS}_{\{F\}_{\tx{reac}}} -
        \underbrace{\int_V[B]^T\{\sigma\}dV}_{\{F\}_{\tx{int}}=[B]\{\sigma\}}=\{0\}
      \end{equation*}
    \end{block}}
    \item<3->[] \fe{Vecteurs de forces nodales équivalentes :}
                   {Equivalent nodal forces vector:}
    \scriptsize
    \begin{tabular}{lll}
      $\{F\}_{\tx{imp}}$  & \fe{forces volumiques $f_{\tx{imp}}$ et surfaciques $t_{\tx{imp}}$ imposées}
                               {imposed volume $f_{\tx{imp}}$ and surface $t_{\tx{imp}}$ forces}          & \kwr{PRES FSUR CNEQ}\\
      $\{F\}_{\tx{reac}}$ & \fe{réactions aux déplacements imposés $u_{\tx{imp}}$}
                               {reactions to imposed displacements $u_{\tx{imp}}$}                        & \kwr{REAC}\\
      $[B]\{\sigma\}$     & \fe{forces volumique intérieures}{internal volume forces}                     & \kwr{BSIG}
    \end{tabular}
  \end{itemize}
\end{frame}

\begin{frame}{\fe{Solveur mécanique}{Mechanical solver}}
             {\fe{Résidu}{Imbalance}}
  \begin{itemize}
    \item \fe{Résidu = mesure du déséquilibre}
             {Imbalance = measure of deviation of equilibrium}
    \begin{equation*}
      \{R\}=\{F\}_{\tx{imp}}+\{F\}_{\tx{reac}}-[B]\{\sigma\}
    \end{equation*}
    \item \fe{Équilibre atteint lorsque}{The balance is obtained when}
    \begin{equation*}
      \|R\| < \varepsilon F_{\tx{ref}}
    \end{equation*}
    \scriptsize
    \begin{tabular}{cl}
      $\|R\|$            & \fe{norme du résidu (par exemple la norme infinie)}
                              {norm of the imbalance vector (for instance infinity norm)}\\
      $\varepsilon$      & \fe{précision du calcul (fournie par utilisateur)}
                              {computational precision (given by user)}\\
      $\{F\}_{\tx{ref}}$ & \fe{effort de référence du problème considéré}
                              {reference force for the studied problem}
    \end{tabular}
  \end{itemize}
\end{frame}

\begin{frame}{\fe{Solveur mécanique}{Mechanical solver}}
             {\fe{Comportement non linéaire}{Non linear behavior}}
  \footnotesize
  \begin{itemize}
    \item<1-> \fe{Décomposition avec matrice de rigidité élastique linéaire}
                 {Decomposition with the linear elastic stiffness matrix}\\
    \begin{tabular}{lrrl}
      $\tod{\sigma}=\mathbb{E}:\tod{\varepsilon}_{\tx{lin}}+\tod{\sigma}_{\tx{nlin}}$ & \fe{avec}{with} & $\tod{\varepsilon}_{\tx{lin}}=$ & $\frac{1}{2}(\nabla u+\nabla^T u)$\\
       & & $\{\varepsilon\}_{\tx{lin}} =$ & $[B]\{U\}$
    \end{tabular}
    \item<2-> \fe{L'équilibre s'écrit alors}{Equilibrium can be re written}
    \begin{equation*}
      [B]\{\sigma\}=\{F\}_{\tx{imp}}+\{F\}_{\tx{reac}}
    \end{equation*}
    \begin{equation*}
      [B][E]\{\varepsilon\}_{\tx{lin}}=\{F\}_{\tx{imp}}+\{F\}_{\tx{reac}}-[B]\{\sigma\}_{\tx{nlin}}
    \end{equation*}
    \begin{equation*}
      \underbrace{\int_V[B]^T[E][B]dV}_{[K]}\{U\} =\{F\}_{\tx{imp}}+\{F\}_{\tx{reac}}-[B]\{\sigma\}_{\tx{nlin}}
    \end{equation*}
    \onslide<3->{
    \begin{equation*}
      [K]\{U\}=\{F\}_{\tx{imp}}+\{F\}_{\tx{reac}}-[B]\{\sigma\}_{\tx{nlin}}
    \end{equation*}\\~\\
    \scriptsize
    \begin{tabular}{llrl}
      $\{U\}$                      & \fe{déplacements nodaux}{nodal displacements}                                      & $u(x)=$                       & $[N(x)]\{U\}$\\
      $\{\varepsilon\}_{\tx{lin}}$ & \fe{déformations linéaires (petites déformations)}{linear strains (small strains)} & $\{\varepsilon\}_{\tx{lin}}=$ & $[B]\{U\}$\\
      $\{\sigma\}_{\tx{nlin}}$     & \fe{contraintes non linéaires}{non linear stresses}                                & & \\
      $[E]$                        & \fe{matrice de Hooke}{Hooke matrix}                                                & \kwr{ELAS} & \\
      $[K]$                        & \fe{matrice de rigidité élastique}{elastic stiffness matrix}                       & \kwr{RIGI} &
    \end{tabular}}
  \end{itemize}
\end{frame}

\begin{frame}{\fe{Solveur mécanique}{Mechanical solver}}
             {\fe{Prise en compte des déplacements imposés}{Introduction of the imposed displacements}}
  \begin{itemize}
    \item<1-> \fe{Matrice de blocage}{Boundary condition matrix}
    \scriptsize
    \begin{tabular}{rl}
      $u=$        & $u_{\tx{imp}}$ \quad \fe{sur}{on} $\partial V^u$\\
      $[A]\{U\}=$ & $\{u\}_{\tx{imp}}$
    \end{tabular}
    \normalsize
    \item<2-> \fe{Multiplicateurs de Lagrange}{Lagrange Multipliers}\\
    \scriptsize
    \fe{depuis un problème linéaire et sans contrainte :}{Ffom a linear and unconstraint problem:}
    $[K]\{U\}=\{F\}_{\tx{imp}}-[B]\{\sigma\}_{\tx{nlin}}$\\
    \fe{on ajoute les inconnues $\{\lambda\}$ :}{we add the unknowns $\{\lambda\}$:}
    \begin{equation*}
      [K]\{U\}+\underbrace{\red{[A]^T\{\lambda\}}}_{-\{F\}_{\tx{reac}}}=\{F\}_{\tx{imp}}-[B]\{\sigma\}_{\tx{nlin}} \quad \tx{\fe{avec}{with}} [A]\{U\}=\{u\}_{\tx{imp}}
    \end{equation*}
    \onslide<3->{
      \begin{block}{\fe{L'équilibre s'écrit alors :}{The equilibrium is now written:}}
        \begin{equation*}
          \underbrace{
            \begin{bmatrix}
              \green{K} & A^T\\
              A         & 0
            \end{bmatrix}}_{[\hat{K}]}
          \begin{Bmatrix}
            U\\
            \lambda
          \end{Bmatrix}
          =
          \begin{Bmatrix}
            \green{F_{\tx{imp}}}-\green{[B]\sigma_{\tx{nlin}}}\\
            u_{\tx{imp}}
          \end{Bmatrix}
        \end{equation*}
      \end{block}
    \scriptsize
    \fe{\begin{tabular}{lll}
          $[A]$              & matrice blocages/relations  & \kwr{BLOQ RELA}\\
          $\{u\}_{\tx{imp}}$ & déplacements nodaux imposés & \kwr{DEPI}\\
          $\{\lambda\}$      & multiplicateurs de Lagrange &
        \end{tabular}}
       {\begin{tabular}{lll}
        $[A]$         & constraints/relations matrix & \kwr{BLOQ RELA}\\
        $\{d\}$       & imposed nodal displacements  & \kwr{DEPI}\\
        $\{\lambda\}$ & Lagrange multipliers         &
      \end{tabular}}}\\
    \scriptsize
    \onslide<4->{
      $\green{[K],~\{F\}_{\tx{imp}},~[B]\{\sigma\}_{\tx{nlin}}}$
      \fe{dépendents de $\{U\}$ \tiny (comportement non lin., grands dépl., forces suiveuses \dots)}
         {depends on $\{U\}$ \tiny (non lin. behavior, large displacements, following forces \dots)}}
  \end{itemize}
\end{frame}


\begin{frame}{\fe{Solveur mécanique}{Mechanical solver}}
             {\fe{Décomposition incrémentale 1/2}{Incremental décomposition 1/2}}
  \begin{itemize}
    \item<1-> \fe{Calcul d'un pas de temps entre $t_0$ et $t_1$}
                 {Computation on a time step between $t_0$ and $t_1$}
    \scriptsize
    \begin{tabular}{lll}
      $\green{t_0~\{U\}_0~\{\lambda\}_0~\{\sigma\}_0}$ & \fe{\green{état connu}}{\green{known state}}                 & \fe{\green{début du pas de temps}}{\green{beginning of time step}}\\
      $  \red{t_1~\{U\}_1~\{\lambda\}_1~\{\sigma\}_1}$ & \fe{\red{état \tou{recherché}}}{\red{\tou{unknown} state}} & \fe{\red{fin du pas de temps}}{\red{end of time step}}\\
    \end{tabular}
    \item<2-> \fe{Décomposition incrémentale des déplacements nodaux}
                 {Incremental decomposition of nodal displacements}\\
    \scriptsize
    \begin{equation*}
      \red{\{U\}_1^i}=\green{\{U\}_0}+\red{\Delta \{U\}_1^i} \qquad \fe{\tx{et}}{\tx{and}} \qquad \red{\Delta \{U\}_1^{i+1}}=\red{\Delta \{U\}_1^i}+\violet{\delta \{U\}_1^{i+1}}
    \end{equation*}
    \fe{et donc}{and so}
    \begin{equation*}
      \red{\{U\}_1^{i+1}}=\red{\{U\}_1^i}+\violet{\delta \{U\}_1^{i+1}}
    \end{equation*}
    \scriptsize
    \begin{tabular}{ll}
      $\red{\{U\}_1^i}$               & \fe{estimation du déplacement $\red{\{U\}_1}$ à l'itération $i$}
                                           {estimation of displacement $\red{\{U\}_1}$ at iteration $i$}\\
      $\red{\Delta \{U\}_1^i}$        & \fe{estimation de l'incrément de déplacement à l'itération $i$}
                                           {estimation of displacement increment at iteration $i$}\\
      $\violet{\delta \{U\}_1^{i+1}}$ & \fe{correction de l'incrément de déplacement à l'itération $i$}
                                           {correction of displacement increment at iteration $i$}
    \end{tabular}
    \item<3-> \fe{Décomposition incrémentale des conditions sur les déplacements}
                 {Incremental decomposition of boundary conditions}
    \scriptsize
    \begin{equation*}
      [A]\red{\{U\}_1^{i+1}}=\{u\}_{\tx{imp},1}
    \end{equation*}
    \begin{equation*}
      [A]\violet{\delta\{U\}_1^{i+1}}=\{u\}_{\tx{imp},1}-[A]\red{\{U\}_1^i}
    \end{equation*}
  \end{itemize}
\end{frame}

\begin{frame}{\fe{Solveur mécanique}{Mechanical solver}}
             {\fe{Décomposition incrémentale 2/2}{Incremental décomposition 2/2}}
  \begin{itemize}
    \item<1-> \fe{Décomposition incrémentale de l'équilibre}
                 {Incremental decomposition of equilibrium}
    \scriptsize
    \begin{equation*}
      [K]\red{\{U\}_1^{i+1}}+[A]^T\red{\{\lambda\}_1^{i+1}} = \{F\}_{\tx{imp},1} - [B]\red{\{\sigma\}_{\tx{nlin},1}^i}
    \end{equation*}
    \begin{equation*}
      [K]\violet{\delta\{U\}_1^{i+1}} = \{F\}_{\tx{imp},1} - [A]^T\red{\{\lambda\}_1^{i+1}} - \underbrace{([K]\red{\{U\}_1^i} + [B]\red{\{\sigma\}_{\tx{nlin},1}^i})}_{[B]\red{\{\sigma\}_1^i}}
    \end{equation*}
    \begin{equation*}
      [K]\violet{\delta\{U\}_1^{i+1}} = \underbrace{\{F\}_{\tx{imp},1} - [A]^T\red{\{\lambda\}_1^{i+1}} - [B]\red{\{\sigma\}_1^i}}_{\red{\{R\}_1^i}}
    \end{equation*}
    \normalsize
    \onslide<2->{
      \begin{block}{\fe{L'équilibre s'écrit finalement}{Equilibrium is finally written}}
        \begin{equation*}
          \underbrace{
            \begin{bmatrix}
              K & A^T\\
              A & 0
            \end{bmatrix}}_{[\hat{K}]}
          \begin{Bmatrix}
            \violet{\delta U_1^{i+1}}\\
            \red{\lambda_1^{i+1}}
          \end{Bmatrix}
          =
          \begin{Bmatrix}
            F_{\tx{imp}}-[B]\red{\sigma_1^i}\\
            u_{\tx{imp},1}-[A]\red{U_1^i}
          \end{Bmatrix}
        \end{equation*}
      \end{block}}
  \end{itemize}
\end{frame}




\fe{\subsection{La procédure UNPAS}}{\subsection{The UNPAS procedure}}
\label{unpas}

\begin{frame}{\fe{Algoritme de minimisation du résidu}{Imbalance minimization algorithm}}
  \begin{itemize}
    \item<1-> \fe{Initialisations}{Initializations}
    \scriptsize
    \begin{tabular}{ll}
      &\\
      $\left[\red{\{U\}_1^{i=0}~\{\lambda\}_1^{i=0}~\{\sigma\}_1^{i=0}}\right] = \left[\green{\{U\}_0~\{\lambda\}_0~\{\sigma\}_0^{\white{0}}}\right]$ & \fe{initialisation de la solution}{solution initialization} \vspace{1mm}\\
      $\red{\{F\}_{\tx{reac},1}^{i=0}} = -[A]^T\red{\{\lambda\}_1^{i=0}}$                                           & \fe{initialisation des réactions \kwr{REAC}}{reactions initialization \kwr{REAC}} \vspace{1mm}\\
      $F_{\tx{ref}} = \|\{F\}_{\tx{imp},1}+\red{\{F\}_{\tx{reac},1}^{i=0}}\|$                                       & \fe{norme de convergence \kwr{MAXI}\kwg{ 'ABS'}}{convergence norm \kwr{MAXI}\kwg{ 'ABS'}} \vspace{1mm}\\
      $\{R\}_1^{i=0} = \{F\}_{\tx{imp},1}+\red{\{F\}_{\tx{reac},1}^{i=0}}-[B]\red{\{\sigma\}_1^{i=0}}$              & \fe{premier résidu \kwr{BSIG}}{first imbalance (residual) \kwr{BSIG}}\\
      &
    \end{tabular}
    \normalsize
    \item<2-> \fe{Tant que : $\|\{R\}_1^i\| \geq \varepsilon F_{\tx{ref}}$}{While: $\|\{R\}_1^i\| \geq \varepsilon F_{\tx{ref}}$}
    \begin{textblock*}{3cm}(9.5cm,-0.5cm)
      \includegraphics[width=3cm]{images/obi_wan}
    \end{textblock*}
    \scriptsize
    \begin{tabular}{ll}
      &\\
      $\left[\violet{\delta\{U\}_1^{i+1}}~\red{\{\lambda\}_1^{i+1}}\right] = [\hat{K}]^{-1}\{R\}_1^i$ & \fe{résolution \kwr{RESO}}{resolution \kwr{RESO}} \vspace{1mm}\\
      $\red{\{U\}_1^{i+1}} = \red{\{U\}_1^i} + \violet{\delta\{U\}_1^{i+1}}$               & \fe{estim. déplacements}{estim. displacements} \vspace{1mm}\\
      $\red{\{\varepsilon\}_1^{i+1}} = \mathcal{D}(\red{\{U\}_1^{i+1}})$                   & \fe{estim. déformations \kwr{EPSI}}{estim. strains \kwr{EPSI}} \vspace{1mm}\\
      $\red{\{\sigma\}_1^{i+1}} = \mathcal{C}(\Delta\{\varepsilon\}_1^{i+1})$              & \fe{estim. contraintes \kwr{COMP}}{estim. stresses \kwr{COMP}} \vspace{1mm}\\
      $\red{\{F\}_{\tx{reac},1}^{i+1}} = -[A]^T\red{\{\lambda\}_1^{i+1}}$                  & \fe{estim. réactions \kwr{REAC}}{estim. reactions \kwr{REAC}} \vspace{1mm}\\
      $\{R\}_1^{i+1} = \{F\}_{\tx{imp},1}+\red{\{F\}_{\tx{reac},1}^{i+1}}-[B]\red{\{\sigma\}_1^{i+1}}$ & \fe{nouveau résidu \kwr{BSIG}}{new imbalance \kwr{BSIG}}
    \end{tabular}
    \normalsize
    \item<2->[] \fe{Fin}{End}
  \end{itemize}
\end{frame}

\begin{frame}{\fe{Algorithme de UNPAS}{UNPAS algorithm}}
  \tiny
  \begin{tikzpicture}[node distance=0.55cm,
    every node/.style={fill=white}, align=center]
  % Cellules (style, position, contenu)
  \fe{\node (ini)   [bstep] {Initialisations, \kwv{CHARMECA}};}
     {\node (ini)   [bstep] {Initializations, \kwv{CHARMECA}};}
  \fe{\node (ini0)  [note, right of=ini, xshift=1.5cm, anchor=west] {Instanciations matériaux, calcul matrice de rigidité $[K]$, forces imposées $\{F\}_{\tx{imp}}$};}
     {\node (ini0)  [note, right of=ini, xshift=1.5cm, anchor=west] {Material updating, stiffness matrix $[K]$ calculation, imposed forces $\{F\}_{\tx{imp}}$};}

  \fe{\node (1r)    [bstep, below of=ini] {1er résidu};}
     {\node (1r)    [bstep, below of=ini] {1st imbalance};}
  \fe{\node (1r0)   [note, right of=1r, xshift=1.5cm, anchor=west] {Déséquilibre entre les forces internes et externes\\
                                                                  $\{R\}_1^{i=0}=\{F\}_{\tx{imp},1}+\green{\{F\}_{\tx{reac},0}}-[B]\green{\{\sigma\}_0}$ \kwr{BSIG}};}
     {\node (1r0)   [note, right of=1r, xshift=1.5cm, anchor=west] {Imbalance between the internal and external forces\\
                                                                  $\{R\}_1^{i=0}=\{F\}_{\tx{imp},1}+\green{\{F\}_{\tx{reac},0}}-[B]\green{\{\sigma\}_0}$ \kwr{BSIG}};}

  \fe{\node (1res)  [rstep, below of=1r, yshift=-0.1cm] {1er \kwr{RESO}\\ norme de convergence};}
     {\node (1res)  [rstep, below of=1r, yshift=-0.1cm] {1st \kwr{RESO}\\ convergence norm};}
  \fe{\node (1res0) [note, right of=1res, xshift=1.5cm, anchor=west] {Incrément déplacement $\red{\Delta\{U\}_1^{i=0}}=\violet{\delta\{U\}_1^{i=0}}=[K]^{-1}\{R\}_1^{i=0}$ \kwr{RESO}\\
                                                                    Norme de convergence $F_{\tx{ref}}$ selon $\{R\}_1^{i=0}$ et $\red{\Delta\{U\}_1^{i=0}}$};}
     {\node (1res0) [note, right of=1res, xshift=1.5cm, anchor=west] {Displacement increment $\red{\Delta\{U\}_1^{i=0}}=\violet{\delta\{U\}_1^{i=0}}=[K]^{-1}\{R\}_1^{i=0}$ \kwr{RESO}\\
                                                                    Convergence norm $F_{\tx{ref}}$ according to $\{R\}_1^{i=0}$ and $\red{\Delta\{U\}_1^{i=0}}$};}

  \fe{\node (acc)  [rstep, below of=1res, yshift=-0.13cm] {Accélération convergence};}
     {\node (acc)  [rstep, below of=1res, yshift=-0.13cm] {Convergence acceleration};}
  \fe{\node (acc0) [note, right of=acc, xshift=1.5cm, anchor=west] {Correction du résidu à partir des 3 résidus précédents \kwr{ACT3}};}
     {\node (acc0) [note, right of=acc, xshift=1.5cm, anchor=west] {Imbalance correction from the 3 previous imbalances \kwr{ACT3}};}

      \node (reso)  [rstep, below of=acc] {\kwr{RESO}};
  \fe{\node (reso0) [note, right of=reso, xshift=1.5cm, anchor=west] {Correction d'incrément $\violet{\delta\{U\}_1^{i+1}} | \red{\{\lambda\}_1^{i+1}}=[K]^{-1}\{R\}_1^i$};}
     {\node (reso0) [note, right of=reso, xshift=1.5cm, anchor=west] {Correction of increment $\violet{\delta\{U\}_1^{i+1}}=[K]^{-1}\{R\}_1^i$};}

  \fe{\node (comp) [rstep, below of=reso, yshift=-0.1cm]  {Intégration comportement\\ \kwv{CHARMECA}};}
     {\node (comp) [rstep, below of=reso, yshift=-0.1cm]  {Integration behavior law\\ \kwv{CHARMECA}};}
  \fe{\node (comp0) [note, right of=comp, xshift=1.5cm, anchor=west] {Incrément déformations $\Delta\{\varepsilon\}_1^{i+1}$, contraintes $\{\sigma\}_1^{i+1}$, variables internes $\{V\}_1^{i+1}$ \kwr{COMP}};}
     {\node (comp0) [note, right of=comp, xshift=1.5cm, anchor=west] {Strains increment $\Delta\{\varepsilon\}_1^{i+1}$, stresses $\{\sigma\}_1^{i+1}$, internal variables $\{V\}_1^{i+1}$ \kwr{COMP}};}

  \fe{\node (res)  [rstep, below of=comp, yshift=-0.13cm]  {Nouveau résidu};}
     {\node (res)  [rstep, below of=comp, yshift=-0.13cm]  {New imbalance};}
  \fe{\node (res0) [note, right of=res, xshift=1.5cm, anchor=west] {Réactions $\red{\{F\}_{\tx{reac},1}^{i+1}} = -[A]^T\red{\{\lambda\}_1^{i+1}}$ \kwr{REAC}\\
                                                                  Résidu $\{R\}_1^{i+1} = \{F\}_{\tx{imp},1}+\red{\{F\}_{\tx{reac},1}^{i+1}}-[B]\red{\{\sigma\}_1^{i+1}}$ \kwr{BSIG}};}
     {\node (res0) [note, right of=res, xshift=1.5cm, anchor=west] {Reactions $\red{\{F\}_{\tx{reac},1}^{i+1}} = -[A]^T\red{\{\lambda\}_1^{i+1}}$ \kwr{REAC}\\
                                                                  Imbalance $\{R\}_1^{i+1} = \{F\}_{\tx{imp},1}+\red{\{F\}_{\tx{reac},1}^{i+1}}-[B]\red{\{\sigma\}_1^{i+1}}$ \kwr{BSIG}};}

  \fe{\node (eq)  [rtest, below of=res, yshift=-0.2cm]  {Équilibre ?};}
     {\node (eq)  [rtest, below of=res, yshift=-0.2cm]  {Equilibrium?};}
  \fe{\node (eq0) [note, right of=eq, xshift=1.5cm, anchor=west] {Mesure normée du résidu $\leq$ critère ? \kwg{'PRECISION' 'FTOL' 'MTOL'}\\
                                                                Variation de l'incrément de déformations entre 2 itérations < critère ?};}
     {\node (eq0) [note, right of=eq, xshift=1.5cm, anchor=west] {Magnitude of the imbalance $\leq$ criterion? \kwg{'PRECISION' 'FTOL' 'MTOL'}\\
                                                                Variation of the strains increment between 2 iterations < criterion?};}

  \fe{\node (nc)  [rtest, below of=eq, yshift=-0.5cm]  {Non convergence ?};}
     {\node (nc)  [rtest, below of=eq, yshift=-0.5cm]  {Non congergence?};}
  \fe{\node (nc0) [note, right of=nc, xshift=1.5cm, anchor=west] {Détection de la non convergence :\\
                                                                - nombre maximal d'itérations atteint\\
                                                                - augmentation significative du résidu sur plusieurs itérations\\
                                                                Passage en convergence forcée};}
     {\node (nc0) [note, right of=nc, xshift=1.5cm, anchor=west] {Non convergence detection:\\
                                                                - maximum number of iterations reached\\
                                                                - significant increase of the imbalance over several iterations\\
                                                                Forced convergence process};}
  \fe{\node (sauv) [bstep, below of=nc, yshift=-0.4cm] {Enregistrement résultats};}
     {\node (sauv) [bstep, below of=nc, yshift=-0.4cm] {Save esults};}

  \fe{\node (conv) [btest, below of=sauv, yshift=-0.2cm] {Convergence ?};}
     {\node (conv) [btest, below of=sauv, yshift=-0.2cm] {Convergence?};}

  \fe{\node (fin)   [bstep, right of=conv, xshift=2cm] {Fin};}
     {\node (fin)   [bstep, right of=conv, xshift=2cm] {End};}

% Connections entre les cellules
  \draw[->]               (ini) -- (1r);
  \draw[->]               (1r) -- (1res);
  \draw[->]               (1res) -- (acc);
  \draw[->, draw=red]     (acc) -- (reso);
  \draw[->, draw=red]     (reso) -- (comp);
  \draw[->, draw=red]     (comp) -- (res);
  \draw[->, draw=red]     (res) -- (eq);
  \fe{\draw[->, draw=red] (eq) -- node[xshift=0.4cm]{non} (nc);}
     {\draw[->, draw=red] (eq) -- node[xshift=0.4cm]{no} (nc);}
  \fe{\draw[->]           (eq.east) -- node[yshift=0.2cm]{oui} ++(0.4,0) -- ++(0,-2) -- (sauv.east);}
     {\draw[->]           (eq.east) -- node[yshift=0.2cm]{yes} ++(0.4,0) -- ++(0,-2) -- (sauv.east);}
  \fe{\draw[->, draw=red] (nc.west) -- node[xshift=0.1cm, yshift=0.2cm]{non} ++(-0.4,0) -- ++(0,3.68) -- (acc.west);}
     {\draw[->, draw=red] (nc.west) -- node[yshift=0.2cm]{no}  ++(-0.4,0) -- ++(0,3.68) -- (acc.west);}
  \fe{\draw[->]           (nc) -- node[xshift=0.4cm, yshift=0.05cm]{oui} (sauv);}
     {\draw[->]           (nc) -- node[xshift=0.4cm]{yes} (sauv);}
  \draw[->]               (sauv) -- (conv);
  \fe{\draw[->] (conv.west) -- node[yshift=0.2cm]{non} ++(-0.8,0) -- ++(0,7.25) -- (ini.west);}
     {\draw[->] (conv.west) -- node[yshift=0.2cm]{no}  ++(-0.8,0) -- ++(0,7.25) -- (ini.west);}
  \fe{\draw[->] (conv.east) -- node[yshift=0.2cm]{oui} (fin.west);}
     {\draw[->] (conv.east) -- node[yshift=0.2cm]{yes} (fin.west);}
  \draw[dotted]           (ini) -- (ini0);
  \draw[dotted]           (1r)  -- (1r0);
  \draw[dotted]           (1res) -- (1res0);
  \draw[dotted]           (acc)  -- (acc0);
  \draw[dotted]           (reso) -- (reso0);
  \draw[dotted]           (comp)  -- (comp0);
  \draw[dotted]           (res) -- (res0);
  \draw[dotted]           (eq) -- (eq0);
  \draw[dotted]           (nc)  -- (nc0);
  \end{tikzpicture}
\end{frame}

{\setbeamerfont{framesubtitle}{size=\tiny}
\begin{frame}{\fe{Exercice 1 : poutre avec force suiveuse}{Exercise 1: beam with following force}}
             {\url{https://www-cast3m.cea.fr/index.php?page=exemples&exemple=formation_pasapas_1_initial}}
  \small
  \begin{itemize}
    \item \fe{Poutre en flexion}{Beam bending}\\
    \scriptsize
    \fe{base encastrée, force \tou{perpendiculaire} à la poutre, déplacement "important"}
       {clamped base, force \tou{perpendicular} to the beam, "large" displacement}\\
    \begin{tikzpicture}
      \draw [-,thick] (0,0) -- (0,3);
      \draw [->, red] (0,3) -- (0.5,3) node (force) [right, red] {$F$};
      \draw [-]       (-0.2,0) -- (0.25,0);
      \draw [-]       (-0.2,-0.1) -- (-0.1,0);
      \draw [-]       (-0.1,-0.1) -- (0,0);
      \draw [-]       (0,-0.1) -- (0.1,0);
      \draw [-]       (0.1,-0.1) -- (0.2,0);
    \end{tikzpicture}
    \animategraphics[controls,loop,poster=last,height=3cm]{10}{images/exo/exo_1_deformee.}{01}{21}
    \onslide<2->{
    \begin{textblock*}{10cm}(8cm,-3.5cm)
      \includegraphics[height=3cm]{images/exo/exo_1_evol}
    \end{textblock*}}
    \normalsize
    \onslide<3->{
    \item \fe{\red{Problème : l'effort est calculé sur la configuration initiale et n'est pas mis à jour}}
             {\red{Problem: the force is calculated on the initial shape and not updated}}
    \item \fe{\green{Objectif : réappliquer l'effort correctement au cours des pas}}
             {\green{Purpose: to correctly apply the force during the calculation}}
    \begin{center}
      \fe{\avous{~À vous de jouer !}}{\avous{~It's up to you!}}
    \end{center}}
  \end{itemize}
\end{frame}
}

{\setbeamerfont{framesubtitle}{size=\tiny}
\begin{frame}{\fe{Exercice 1 : poutre avec force suiveuse}{Exercise 1: beam with following force}}
             {\url{https://www-cast3m.cea.fr/index.php?page=exemples&exemple=formation_pasapas_1_initial}}
  \begin{itemize}
    \item \fe{Quelques objets utiles}{Some useful objects}\\
    \small
    \kw{p2 } \fe{point au sommet de la poutre, où est appliqué la force}{point at the top of the beam, where the force is prescribed}\\
    \kw{ev1} \fe{évolution de l'amplitude de la force à appliquer vs. temps}{force magnitude to be applied as a function of time}
    \normalsize
    \item \fe{Quelques opérateurs utiles}{Some useful operators}\\
    \small
    \kwr{EXTR} \fe{pour extraire des valeurs d'un champ}{to extract values from a field}\\
    \kwr{COS SIN} \fe{pour faire un peu de trigonométrie}{for a bit of trigonometry}\\
    \kwr{IPOL} \fe{pour interpoler l'amplitude de la force à l'instant de calcul}
                  {to interpolate the force magnitude at the current time step}\\
    \kwr{FORC} \fe{pour appliquer une force ponctuelle}{to apply a point force}
    \normalsize
    \item \fe{Quelques indices utiles de la table}{Some useful table indices}\\
    \small
    \kwg{'ESTIMATION'.'DEPLACEMENTS'} \fe{dernier champ de déplacement convergé}{last converged displacement field}
    \kwg{'WTABLE'.'CHARGEMENT'} \fe{chargement courant}{current load}
    \normalsize
  \end{itemize}
\end{frame}
}

\begin{frame}{\fe{Exercice 1 : poutre avec force suiveuse}{Exercise 1: beam with following force}}
             {\fe{Solution avec PERSO1}{Solution with PERSO1}}
  \footnotesize
  \begin{itemize}
    \item \fe{Activer l'option \kwg{'GRANDS\_DEPLACEMENTS'}\\ (équilibre vérifié sur config. déformée)}
             {Use option \kwg{'GRANDS\_DEPLACEMENTS'}\\ (equilibrium verified on the deformed shape)}
    \item \fe{Utiliser la procédure \kwv{PERSO1} \red{(1 appel / pas de temps)} pour mettre à jour la force\\
              sur la configuration déformée au \tou{début du pas (explicite)}}
             {Use procedure \kwv{PERSO1} \red{(1 call / time step)} to update the force on deformed shape\\
              at \tou{beginning of time step (explicit)}}
    \item \fe{Créer un objet CHARGEMEnt et écraser \kwg{'WTABLE' . 'CHARGEMENT'}}
             {Create an load (CHARGEMEnt object) and overwrite \kwg{'WTABLE' . 'CHARGEMENT'}}
  \end{itemize}
  \vspace{4.5cm}
  \scriptsize
  \begin{textblock*}{10cm}(0.3cm,-2.9cm)
    \fe{\emph{Programme principal}}{\emph{Main program}}
    \lstinputlisting[basicstyle=\ttfamily\tiny, language=gibiane, firstline=93, lastline=97]{dgibi/formation_pasapas_1_solution.dgibi}
  \end{textblock*}
  \begin{textblock*}{10cm}(6.3cm,-4.2cm)
    \fe{\emph{\violet{Procédure PERSO1}}}{\emph{\violet{PERSO1 procedure}}}
    \lstinputlisting[basicstyle=\ttfamily\tiny, language=gibiane, firstline=53, lastline=67]{dgibi/formation_pasapas_1_solution.dgibi}
  \end{textblock*}
\end{frame}

\begin{frame}{\fe{Exercice 1 : poutre avec force suiveuse}{Exercise 1: beam with following force}}
             {\fe{Solution avec PERSO1}{Solution with PERSO1}}
  \small
  \begin{itemize}
    \item \fe{Résultats}{Results}\\
    \scriptsize
    \fe{base encastrée, force \tou{perpendiculaire} à la poutre, déplacement "important"}
       {clamped base, force \tou{perpendicular} to the beam, "large" displacement}\\
    \begin{tikzpicture}
      \draw [-,thick] (0,0) -- (0,3);
      \draw [->, red] (0,3) -- (0.5,3) node (force) [right, red] {$F$};
      \draw [-]       (-0.2,0) -- (0.25,0);
      \draw [-]       (-0.2,-0.1) -- (-0.1,0);
      \draw [-]       (-0.1,-0.1) -- (0,0);
      \draw [-]       (0,-0.1) -- (0.1,0);
      \draw [-]       (0.1,-0.1) -- (0.2,0);
    \end{tikzpicture}
    \hspace{0.9cm}
    \animategraphics[controls,loop,poster=last,height=3cm]{10}{images/exo/exo_1_solu_deformee.}{01}{21}
    \begin{textblock*}{10cm}(8cm,-3.5cm)
      \includegraphics[height=3cm]{images/exo/exo_1_solu_evol}
    \end{textblock*}
    \normalsize
  \end{itemize}
\end{frame}

\begin{frame}{\fe{Exercice 1 : poutre avec force suiveuse}{Exercise 1: beam with following force}}
             {\fe{Solution (bis) avec CHARMECA}{Solution (bis) with CHARMECA}}
  \footnotesize
  \begin{itemize}
    \item \fe{Idem mais avec la procédure \kwv{CHARMECA}, pour mettre à jour la force\\
              sur la configuration déformée au \tou{début du pas (explicite)}}
             {Idem but with the \kwv{CHARMECA} procedure, to update the force on deformed shape\\
              at \tou{beginning of time step (explicit)}}
    \item \fe{Pas besoin de CHARGEMEnt initial}{No need for initial load (CHARGEMEnt object)}
    \item \fe{\red{Plus long (1 appel / itération / pas de temps)} et \green{résultats identiques}}
             {\red{Longer (1 call / each iteration / time step)} and \green{results are identical}}
  \end{itemize}
  \vspace{4.5cm}
  \scriptsize
  \begin{textblock*}{10cm}(0.3cm,-2.9cm)
    \fe{\emph{Programme principal}}{\emph{Main program}}
    \lstinputlisting[basicstyle=\ttfamily\tiny, language=gibiane, firstline=143, lastline=149]{dgibi/formation_pasapas_1_solution.dgibi}
  \end{textblock*}
  \begin{textblock*}{10cm}(6.3cm,-4.3cm)
    \fe{\emph{\violet{Procédure CHARMECA}}}{\emph{\violet{CHARMECA procedure}}}
    \lstinputlisting[basicstyle=\ttfamily\tiny, language=gibiane, firstline=69, lastline=83]{dgibi/formation_pasapas_1_solution.dgibi}
  \end{textblock*}
\end{frame}

\begin{frame}{\fe{Exercice 1 : poutre avec force suiveuse}{Exercise 1: beam with following force}}
             {\fe{Solution (ter) avec CHARMECA}{Solution (ter) with CHARMECA}}
  \footnotesize
  \begin{itemize}
    \item \fe{Idem mais avec la procédure \kwv{CHARMECA}, pour mettre à jour la force\\
              sur la configuration déformée à la \tou{fin du pas (implicite)}}
             {Idem but with the \kwv{CHARMECA} procedure, to update the force on deformed shape\\
              at the \tou{end of time step (implicit)}}
    \item \fe{Attention : peut être instable !}{Watch out: can become unstable!}
  \end{itemize}
  \vspace{4.5cm}
  \scriptsize
  \begin{textblock*}{10cm}(0.3cm,-2.9cm)
    \fe{\emph{Programme principal}}{\emph{Main program}}
    \lstinputlisting[basicstyle=\ttfamily\tiny, language=gibiane, firstline=143, lastline=149]{dgibi/formation_pasapas_1_solution.dgibi}
  \end{textblock*}
  \begin{textblock*}{10cm}(6.3cm,-4.4cm)
    \fe{\emph{\violet{Procédure CHARMECA}}}{\emph{\violet{CHARMECA procedure}}}
    \lstinputlisting[basicstyle=\ttfamily\tiny, language=gibiane, firstline=151, lastline=168]{dgibi/formation_pasapas_1_solution.dgibi}
  \end{textblock*}
\end{frame}

{\setbeamerfont{framesubtitle}{size=\tiny}
\begin{frame}{\fe{Exercice 2 : fissuration par suppression d'éléments}
                 {Exercise 2: fracture by elements removal}}
             {\url{https://www-cast3m.cea.fr/index.php?page=exemples&exemple=formation_pasapas_2_initial}}
  \small
  \begin{itemize}
    \item \fe{Plaque en traction \textcolor{green}{(déplacements imposés)}}
             {Plate under tension \textcolor{green}{(imposed displacements)}}\\
    \includegraphics[height=3cm]{images/exo/exo_2_chargement} \hspace{0.2cm}
    \animategraphics[controls,loop,poster=last,height=3.5cm]{7}{images/exo/exo_2_sigma.}{01}{11} \hspace{0.2cm}
    \includegraphics[height=3cm]{images/exo/exo_2_evol}
    \item<2-> \fe{\green{Objectif : modéliser la rupture en \tou{supprimant les éléments} au cours du calcul}\\
                  On utilisera un critère simple sur la 1ère contrainte principale :
                  \begin{center}
                    rupture si $\sigma_I \geq$ 22~GPa\\ \avous{~À vous de jouer !}
                  \end{center}}
                 {\green{Purpose: to model fracture by \tou{removing elements} during calculations}\\
                  We will use a simple criterion based on the 1st principal stress:
                  \begin{center}
                    fracture if $\sigma_I \geq$ 22~GPa\\ \avous{~It's up to you!}
                  \end{center}}
  \end{itemize}
\end{frame}
}

{\setbeamerfont{framesubtitle}{size=\tiny}
\begin{frame}{\fe{Exercice 2 : fissuration par suppression d'éléments}
                 {Exercise 2: fracture by elements removal}}
             {\url{https://www-cast3m.cea.fr/index.php?page=exemples&exemple=formation_pasapas_2_initial}}
  \begin{itemize}
    \item \fe{Quelques opérateurs utiles}{Some useful operators}\\
    \small
    \kwr{PRIN} \fe{calcul des contraintes principales}{compute principal stresses}\\
    \kwr{ELEM} \fe{séléction des éléments où un champ vérifie une condition}{select the elements of a field that meets a criteria}\\
    \kwr{REDU} \fe{réduction d'un modèle sur un sous maillage}{reduction of a model on a sub part of a mesh}\\
    \kwr{CHAN} \fe{changement des points support d'un champ}{change the support points of a field}
    \normalsize
    \item \fe{Quelques indices utiles de la table}{Some useful table indices}\\
    \small
    \kwg{'ESTIMATION'.'CONTRAINTES'} \fe{dernier champ de contraintes convergé}{last converged stress field}
    \kwg{'WTABLE'.'MODELE'} \fe{modèle courant}{current model}\\
    \kwg{'WTABLE'.'CARACTERISTIQUES'} \fe{paramètres matériau}{material properties}
    \normalsize
  \end{itemize}
\end{frame}
}

\begin{frame}{\fe{Exercice 2 : fissuration par suppression d'éléments}
                 {Exercise 2: fracture by elements removal}}
             {\fe{Solution avec PERSO1}{Solution with PERSO1}}
  \footnotesize
  \begin{itemize}
    \small
    \item \fe{Utiliser la procédure \kwv{PERSO1}}{Use procedure \kwv{PERSO1}}
    \item \fe{Extraire le modèle et les contraintes dans \kwg{'ESTIMATION'}}{Extract the model and stress field in \kwg{'ESTIMATION'}}
    \item \fe{Calculer la contrainte principale $\sigma_I$}{Compute the 1st principal stress $\sigma_I$}
    \item \fe{Extraire les éléments "intacts"}{Extract the "intact" elements}
    \item \fe{Réduire le modèle sur ce maillage et écraser \kwg{'WTABLE'.'MODELE'}}
             {Reduce the model on this mesh and overwrite \kwg{'WTABLE'.'MODELE'}}
    \item \fe{Réduire les pas de temps}{Reduce time steps}
  \end{itemize}
  \vspace{4.5cm}
  \scriptsize
  \begin{textblock*}{10cm}(0.3cm,-3.2cm)
    \fe{\emph{Programme principal}}{\emph{Main program}}
    \lstinputlisting[basicstyle=\ttfamily\tiny, language=gibiane, firstline=92, lastline=95]{dgibi/formation_pasapas_2_solution.dgibi}
  \end{textblock*}
  \begin{textblock*}{10cm}(6.2cm,-4cm)
    \fe{\emph{\violet{Procédure PERSO1}}}{\emph{\violet{PERSO1 procedure}}}
    \lstinputlisting[basicstyle=\ttfamily\tiny, language=gibiane, firstline=75, lastline=84]{dgibi/formation_pasapas_2_solution.dgibi}
  \end{textblock*}
\end{frame}

\begin{frame}{\fe{Exercice 2 : fissuration par suppression d'éléments}
                 {Exercise 2: fracture by elements removal}}
             {\fe{Solution avec PERSO1}{Solution with PERSO1}}
  \footnotesize
  \begin{itemize}
    \item \fe{Résultats}{Results}\\
    \hspace{1cm}
    \animategraphics[controls,loop,poster=last,height=4.5cm]{7}{images/exo/exo_2_solu_sigma.}{01}{47} \hspace{1.5cm}
    \includegraphics[height=3.5cm]{images/exo/exo_2_solu_evol}
    \item \fe{Modèle peu robuste\\ résultats très sensibles à la discrétisation espace/temps}
             {Undependable model\\ results quite sensitive to time/space discretization}
  \end{itemize}
\end{frame}

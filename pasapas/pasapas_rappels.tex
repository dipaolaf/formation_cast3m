%%%%%%%%%%%%%%%%%%%%%%%%%%%%%%%%%%%%%%%%%%%
\fe{\section{Rappels}}{\section{Reminders}}
\label{rappels}
%%%%%%%%%%%%%%%%%%%%%%%%%%%%%%%%%%%%%%%%%%%

\begin{frame}{\fe{Cast3M, quid ?}{What is Cast3M?}}
  \begin{center}
    \fe{Logiciel de simulation utilisant la \g{méthode des éléments finis} en \g{mécanique/thermique} des \g{structures} et des \g{fluides}}
       {A simulation software using the \g{finite element method} in \g{thermal and mechanical} analysis of \g{structures} and \g{fluids}}\pause
  \end{center}
  \begin{itemize}
    \item \fe{Résolution d'\g{équations aux dérivées partielles}}
             {\g{Partial differential equations} solver}\pause
    \item \fe{\g{Système complet} : solveur, pré/post-processeur, visualisation, import/export des données\dots}
             {\g{Complete software}: solver, pre-processing and post-processing, visualization, reading/writing data\dots}\\~\\
    \begin{center}
      \includegraphics[height=0.25\textheight]{images/plaque.1} \:
      \includegraphics[height=0.25\textheight]{images/plaque.2} \:
      \includegraphics[height=0.25\textheight]{images/plaque.3}
    \end{center}\pause
    \item \fe{Basé sur un langage de commande : \g{Gibiane} (orienté objet)}
             {Based on a programming language: \g{Gibiane} (objet-oriented)}\\
  \end{itemize}
\end{frame}

\begin{frame}{\fe{Nombreux domaines d'application}{Wide range of applications}}
  \small
  \begin{itemize}
    \item<1->\fe{\g{Mécanique des structures}}{\g{Structural mechanics}}\\
    \footnotesize
    \fe{\red{Quasi-statique} (non linéarités matériau, géométrie, conditions limites)}
       {\red{Quasi-static} (non linear behavior, geometry, boundary conditions)}\\
    \onslide<1>{
      \begin{textblock*}{4cm}(0cm,0cm)
        \begin{center}
          \includegraphics[height=0.4\textheight]{images/polycristal}\\
          \tiny{\fe{\emph{Microstructure : agrégat polycristallin}}{\emph{Microstructure: polycrystalline aggregate}}}
        \end{center}
      \end{textblock*}
      \begin{textblock*}{4cm}(3.4cm,0.8cm)
        \begin{center}
          \includegraphics[height=0.4\textheight]{images/tube}\\
          \tiny{\fe{\emph{Composant : liaison d'un tube}}{\emph{Component: connection on a tube}}}
        \end{center}
      \end{textblock*}
      \begin{textblock*}{5cm}(7.2cm,1.6cm)
        \begin{center}
          \includegraphics[height=0.4\textheight]{images/galatee}\\
          \tiny{\fe{\emph{Batiment en béton armé (S. Durand)}}{\emph{Reinforced concrete building (S. Durand)}}}
        \end{center}
      \end{textblock*}
      }
    \onslide<2->\fe{\orange{Contact/frottement}, \green{Flambage}}
                    {\orange{Contact/friction}, \green{Buckling}}\\
    \onslide<2>{
      \begin{textblock*}{10cm}(1cm,0.2cm)
        \begin{center}
          \includegraphics[height=0.25\textheight]{images/sac_a_dos.15}
          \includegraphics[height=0.25\textheight]{images/sac_a_dos.32}
          \includegraphics[height=0.25\textheight]{images/sac_a_dos.41}\\
          \tiny{\fe{\emph{Attachement d'un clip (contact + plasticité)}}{\emph{Locking a clip buckle (contact + plasticity)}}}
        \end{center}
      \end{textblock*}
      }
    \onslide<3->\fe{\blue{Dynamique} (temporelle, modale, interaction fluide structure)}
                   {\blue{Dynamic} (temporal, modal, fluid structure interaction)}\\
    \onslide<3>{
      \begin{textblock*}{10cm}(1cm,0.2cm)
        \begin{center}
          \includegraphics[height=0.2\textheight]{images/cymbale_mode_1}
          \includegraphics[height=0.2\textheight]{images/cymbale_mode_2}
          \includegraphics[height=0.2\textheight]{images/cymbale_mode_4}\\
          \tiny{\fe{\emph{Premiers modes propres d'une cymbale}}{\emph{First eigenmodes of a cymbal}}}
        \end{center}
      \end{textblock*}
      }
    \onslide<4->{\fe{\violet{Rupture} (XFEM, propagation dynamique, zones cohésives)}
                    {\violet{Fracture} (XFEM, dynamic propagation, cohesive zones models)}}\\
    \only<4>{
      \begin{textblock*}{11cm}(1cm,0cm)
        \begin{center}
          \if \animation 1
            \animategraphics[controls,loop,poster=last,height=4cm]{3}{images/rupture_ct/rupture_ct_}{1}{8}
            \animategraphics[controls,loop,poster=last,height=4cm]{3}{images/rupture_ct/rupture_ct_detail_}{1}{8}\\
          \else
            \includegraphics[height=4cm]{images/rupture_ct/rupture_ct_8}
            \includegraphics[height=4cm]{images/rupture_ct/rupture_ct_detail_8}\\
          \fi
          \tiny{\fe{\emph{Rupture d'éprouvette CT, plasticité/endommagement, suppression d'éléments lors du calcul (S. Kebiri)}}
                   {\emph{Fracture of a CT sepcimen, plasticity/dammage, elements removal during computation (S. Kebiri)}}}
        \end{center}
      \end{textblock*}
      }
    \small
    \item<5->\fe{\g{Thermique}}{\g{Thermal analysis}}\\
    \footnotesize
    \onslide<5->{\fe{Conduction, convection, advection, rayonnement, changement de phase}
                    {Conduction, convection, advection, radiation, phase transition}}\\
    \onslide<5>{
      \begin{textblock*}{10cm}(1cm,0.1cm)
        \begin{center}
          \includegraphics[height=0.4\textheight]{images/te_temperature}\hspace{1cm}
          \includegraphics[height=0.4\textheight]{images/te_sigma}\\
          \tiny{\fe{\emph{Thermo mécanique d'un té de tuyauterie}}{\emph{Thermo mechanical analysis of a pipe tee}}}
        \end{center}
      \end{textblock*}
      }
    \small
    \item<6-> \fe{\g{Mécanique des fluides}}{\g{Fluid mechanics}}
    \item<6-> \fe{\g{Diffusion} multi espèces (loi de Fick)}{Multi species \g{diffusion} (Fick's law)}\\
    \item<6-> \fe{Fabrication additive, Métallurgie}{Additive manufacturing, Metallurgy}\\
    \footnotesize
    \only<6>{
      \begin{textblock*}{7cm}(5.7cm,-2cm)
        \begin{center}
          \if \animation 1
            \animategraphics[controls,loop,poster=last,width=5cm]{20}{images/fab_add/fab_add_}{001}{397}\\
          \else
            \includegraphics[width=5cm]{images/fab_add/fab_add_397}\\
          \fi
          \tiny{\fe{\emph{Proportion de bainite lors d'une fabrication additive (C. Berthinier)}}
                   {\emph{Proportion of bainite during the additive manufacturing (C. Berthinier)}}}
        \end{center}
      \end{textblock*}
      }
    \onslide<7>{
      \begin{textblock*}{5cm}(6.5cm,-1.5cm)
        \begin{center}
          \includegraphics[width=5cm]{images/soudage}\\
          \tiny{\fe{\emph{Simulation magnéto thermo hydrodynamique du soudage TIG (arc plasma + bain) (C. Nahed)}}
                   {\emph{Magneto thermo hydro dynamic simulation of TIG welding (plasma arc + weld pool) (C. Nahed)}}}
        \end{center}
      \end{textblock*}
      }
    \small
    \item<8-> \fe{Magnétostatique}{Magneto-statics}
    \item<8-> \fe{Couplage thermo-hygro-mécanique}{Thermo-hygro-mechanics coupling}
    \item<8-> \fe{Optimisation topologique}{Topology optimization}
    \only<8>{
      \begin{textblock*}{5cm}(6.2cm,-6cm)
        \begin{center}
          \if \animation 1
            \animategraphics[controls,loop,poster=last,width=6cm]{10}{images/topoptim/topoptim.}{001}{100}\\~\\
          \else
            \includegraphics[width=6cm]{images/topoptim/topoptim.100}\\~\\
          \fi
          \includegraphics[width=5cm]{images/topoptim/toposurf}\\
          \tiny{\fe{\emph{Optimisation topologique d'un pont}}{\emph{Topology optimization of a bridge}}}
        \end{center}
      \end{textblock*}
      }
  \end{itemize}
\end{frame}

\begin{frame}{\fe{Présentation de PASAPAS}{The PASAPAS procedure}}
  \begin{itemize}
    \item \fe{Objectif}{Objective}\\
    \footnotesize
    \fe{Résolution de problèmes \emph{non linéaires évolutifs} de manière incrémentale\\
        en \red{thermique} et en \blue{mécanique}\\
        le temps peut être physique (ex~: thermique transitoire)\\
        ou non (ex~: plasticité avec chargement progressif)\\
        $\rightarrow$ on parle donc volontiers de \emph{variable d'évolution}\\}
       {Incremental solving of \emph{non linear progressive}\\
        \red{thermal} and \blue{mechanical} problems\\
        Time can be physical (e.g. thermal transients)\\
        or not (e.g. plasticity with progressive loading)\\
        $\rightarrow$ time or pseudo-time is called the \emph{evolution parameter}\\}
    ~
    \normalsize
    \item \fe{Types de non linéarités traitées}{Non linear phenomena considered}\\
    \fe{\red{Comportement} \footnotesize (plasticité, endommagement, matériaux variables, etc.) \normalsize\\
        \orange{Géométrie} \footnotesize (grands déplacements) \normalsize\\
        \green{Déformations} \footnotesize (grandes rotations) \normalsize\\
        \violet{Conditions limites} \footnotesize (rayonnement, frottement, pression suiveuse, etc.) \normalsize}
       {\red{Behavior} (plasticity, damage, variable material properties, etc.)\\
        \orange{Geometry} (large displacements)\\
        \green{Strains} (large rotations)\\
        \violet{Boundary conditions} (radiation, friction, following pressure, etc.)}
  \end{itemize}
\end{frame}

\begin{frame}{\fe{Utilisation de PASAPAS}{PASAPAS use}}
  \begin{itemize}
    \item \fe{\g{Créer une table} contenant toutes les données du problème}
             {\g{Create a table} containing all the data:}\\
    \lstinputlisting[language=gibiane, firstline=1, lastline=9]{dgibi/exemples.dgibi}
    \item \fe{\g{Appeler la procédure}}{\g{Procedure call:}}\\
    \lstinputlisting[language=gibiane, firstline=11, lastline=11]{dgibi/exemples.dgibi}
    \item \fe{\g{Post traitement} des résultats}{Results \g{post-processing}}
  \end{itemize}
\end{frame}

\begin{frame}{\fe{Aperçu des paramètres d'entrée}{Overview of input parameters}}
  \begin{itemize}
    \item \fe{Généralités}{General}
  \end{itemize}
  \tiny
  \hspace{0.4cm}
  \begin{tabular}{lll}
    \kwg{MODELE}           & MMODEL   & \fe{Équations à résoudre, formulation EF (\kwr{MODE})}
                                           {Equations to solve, FE formulation (\kwr{MODE})}\\
    \kwg{CARACTERISTIQUES} & MCHAML   & \fe{Paramètres matériau et/ou géométriques (\kwr{MATE})}
                                           {Material and/or geometrical parameters (\kwr{MATE})}\\
    \kwg{CHARGEMENT}       & CHARGEME & \fe{Évolution des CL et chargements au cours du calcul (\kwr{CHAR})}
                                           {BC and loading variation during calculation (\kwr{CHAR})}
  \end{tabular}
  \normalsize
  \begin{itemize}
    \item \fe{Thermique}{Thermal analysis}
  \end{itemize}
  \tiny
  \hspace{0.4cm}
  \begin{tabular}{lll}
    \kwg{BLOCAGES\_THERMIQUES} & RIGIDITE & \fe{Matrice de blocage des CL de type Dirichlet (\kwr{BLOQ,RELA})}
                                               {Stiffness matrix for Dirichlet BC (\kwr{BLOQ,RELA})}\\
    \kwg{CELSIUS}              & LOGIQUE  & \fe{\kw{= VRAI} si les températures sont en degrés Celsius}
                                               {\kw{= VRAI} (true) if temperature unit is Celsius}\\
    \kwg{TEMPERATURES}\kw{.0}  & CHPOINT  & \fe{Conditions initiales}
                                               {Initial conditions}
  \end{tabular}
  \normalsize
  \begin{itemize}
    \item \fe{Mécanique}{Mechanics}
  \end{itemize}
  \tiny
  \hspace{0.4cm}
  \begin{tabular}{lll}
    \kwg{BLOCAGES\_MECANIQUES}              & RIGIDITE & \fe{Matrice de blocage des CL de type Dirichlet (\kwr{BLOQ,RELA})}
                                                            {Stiffness matrix for Dirichlet BC (\kwr{BLOQ,RELA})}\\
    \kwg{GRANDS\_DEPLACEMENTS}              & LOGIQUE  & \fe{Équilibre vérifié sur les configurations déformées}
                                                            {Equilibrium checked on the deformed mesh}\\
    \kwg{DEPLACEMENTS}\kw{.0}               & CHPOINT  & \fe{Conditions initiales}
                                                            {Initial conditions}\\
    \kwg{CONTRAINTES}\kw{.0}                & MCHAML   & \fe{Idem}{Idem}\\
    \kwg{VARIABLES\_INTERNES}\kw{.0}        & MCHAML   & \fe{Idem}{Idem}\\
    \kwg{DEFORMATIONS\_INELASTIQUES}\kw{.0} & MCHAML   & \fe{Idem}{Idem}
  \end{tabular}
  \normalsize
  \begin{itemize}
    \item \fe{Mécanique (dynamique)}{Mechanics (dynamics)}
  \end{itemize}
  \tiny
  \hspace{0.4cm}
  \begin{tabular}{lll}
    \kwg{DYNAMIQUE}             & LOGIQUE  & \fe{\kw{= VRAI} si calcul dynamique}
                                                {\kw{= VRAI} (true) for dynamics calculations}\\
    \kwg{AMORTISSEMENT}         & RIGIDITE & \fe{Matrice d'amortissement}
                                                {Damping matrix}\\
    \kwg{VITESSES}\kw{.0}       & MCHAML   & \fe{Conditions initiales}
                                                {Initial conditions}\\
    \kwg{ACCELERATIONS}\kw{.0 } & CHPOINT  & \fe{Idem}{Idem}
  \end{tabular}
  \normalsize
  \begin{itemize}
    \item \fe{Instants de calcul et sauvegarde}{Calculation and saving times}
  \end{itemize}
  \tiny
  \hspace{0.4cm}
  \begin{tabular}{lll}
    \kwg{TEMPS\_CALCULES} & LISTREEL & \fe{Liste des instants de calcul (\kwr{PROG})}
                                          {List of times for which results are computed (\kwr{PROG})}\\
    \kwg{TEMPS\_SAUVES}   & LISTREEL & \fe{Liste des instants pour lesquels les résultats sont conservés (\kwr{PROG})}
                                          {List of times for which results are saved (\kwr{PROG})}
  \end{tabular}
\end{frame}

\begin{frame}{\fe{Aperçu des paramètres de sortie}{Overview of ouput parameters}}
  \begin{itemize}
    \item \fe{Résultats}{Results}
  \end{itemize}
  \tiny
  \hspace{0.4cm}
  \begin{tabular}{lll}
    \kwg{TEMPS}                      & TABLE & \fe{Instants de calcul, identiques aux \kwg{TEMPS\_SAUVES}}
                                                  {Time values, identical to \kwg{TEMPS\_SAUVES}}\\
                                     &       & \\
    \kwg{TEMPERATURES}               & TABLE & \fe{Champs solutions pour chaque \kwg{TEMPS\_SAUVES}}
                                                  {Fields (solution) calculated for each stored time \kwg{TEMPS\_SAUVES}}\\
    \kwg{PROPORTION\_PHASE}          & TABLE & \fe{Idem}{Idem}\\
                                     &       & \\
    \kwg{DEPLACEMENTS}               & TABLE & \fe{Idem}{Idem}\\
    \kwg{REACTIONS}                  & TABLE & \fe{Idem}{Idem}\\
    \kwg{CONTRAINTES}                & TABLE & \fe{Idem}{Idem}\\
    \kwg{DEFORMATIONS\_INELASTIQUES} & TABLE & \fe{Idem}{Idem}\\
    \kwg{VARIABLES\_INTERNES}        & TABLE & \fe{Idem}{Idem}\\
    \kwg{VITESSES}                   & TABLE & \fe{Idem}{Idem}\\
    \kwg{ACCELERATIONS}              & TABLE & \fe{Idem}{Idem}
  \end{tabular}
  \normalsize
\end{frame}

\begin{frame}{\fe{Exemples de post traitement}{Post processing examples}}
  \begin{itemize}
    \item \fe{Extraction des champs solution :}{Solution fields extraction:}\\
      \fe{avec l'indice dans la table}{from the table index}\\
      \lstinputlisting[language=gibiane, firstline=13, lastline=13]{dgibi/exemples.dgibi}
      \fe{ou bien avec l'instant de calcul}{or from the time value}\\
      \lstinputlisting[language=gibiane, firstline=14, lastline=14]{dgibi/exemples.dgibi}
    \item \fe{Tracé en mode graphique interactif :}{Graphical mode, interactive plot:}\\
    \lstinputlisting[language=gibiane, firstline=15, lastline=15]{dgibi/exemples.dgibi}
    \item \fe{Évolution temporelle d'un champ calculé :}{Evolution of calculated field with time:}\\
    \lstinputlisting[language=gibiane, firstline=16, lastline=16]{dgibi/exemples.dgibi}
  \end{itemize}
\end{frame}

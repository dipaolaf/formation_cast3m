%%%%%%%%%%%%%%%%%%%%%%%%%%%%%%%%%%%%%%%%%%%%%%%%%%%%%%%%%%
\fe{\section{Langage Gibiane}}{\section{Gibiane Language}}
\label{gibiane}
%%%%%%%%%%%%%%%%%%%%%%%%%%%%%%%%%%%%%%%%%%%%%%%%%%%%%%%%%%

\begin{frame}{\fe{Le langage Gibiane}{The Gibiane language}}
  \begin{itemize}
    \item \fe{\red{Langage de programmation} destiné au calcul EF}
             {\red{Programming language} dedicated to FE calculation}\\
    \footnotesize
    \begin{textblock*}{5cm}(7cm,0.5cm)
      \includegraphics[width=3cm]{images/gibi_genie}
    \end{textblock*}
    \fe{Objets classiques (entiers, flottants, chaines, logiques, tables)\\
        Instructions conditionnelles\\
        Boucles itératives\\
        Sous structuration\\
        Récursivité}
       {Classical objects (integer, floating-point, string, logical , tables)\\
       Flow control statements\\
       Loops and iterations\\
       Subroutines\\
       Recursion}
    \normalsize
    \item \fe{Langage \red{interprété}}{\red{Interpreted} language}\\
    \footnotesize
    \fe{Le programme peut être exécuté dès que le script est modifié\\
        Le programme peut être exécuté en mode interactif}
       {You can run the program as soon as you make changes to the file\\
       You can run it in a interactive mode}
    \normalsize
    \item \fe{Langage \red{orienté objet}}{\red{Object-oriented} language}\\
    \footnotesize
    \fe{Tout est traité comme un objet\\
        Pas besoin de déclarer les variables ou de spécifier leur type}
       {Everything in the program is treated as an object\\
       No need to declare variables or to specify the type of a variable}
    \normalsize
    \item \fe{Mots clefs en français}{French key words}
    \item \fe{Programmation facile et rapide}{Easy to learn, easy to read}
  \end{itemize}
\end{frame}

\begin{frame}{\fe{Gibiane : règles de syntaxe}{Gibiane: syntax rules}}
  \begin{itemize}
    \item \fe{Ligne(s) de commande}{Statements lines}\\
    \footnotesize
    \fe{\red{500 caractères max} par instruction\\
        Une instruction peut être écrite sur plusieurs lignes\\
        Se \red{termine par un point virgule \kw{;}}\\
        Le \red{symbole d'affectation} est le signe égal \kwr{=}}
       {\red{500 characters max} per statement\\
        A statement can be written on several lines\\
        \red{End with a semicolon \kw{;}}\\
        The \red{assignment operator} is the equals sign \kwr{=}}
    \normalsize
    \item \fe{Insensibilité à la casse}{Case insensitive}\\
    \footnotesize
    \kw{TOTO = 3.14 ;}\\
    \kw{A~~~ = 2.~}\kwr{*}\kw{~tOTo ;}
    \fe{la variable \kw{A} vaut bien 6.28}{variable \kw{A} has the value of 6.28}\\
    \normalsize
    \item \fe{Fin du fichier de données}{Program ends}\\
    \footnotesize
    \fe{Commande \kwr{FIN ;} $\Rightarrow$ arrêt de Cast3M}
       {Statement \kwr{FIN ;} $\Rightarrow$ Cast3M stop}\\
    \fe{Ligne vide ou un EOF $\Rightarrow$ mode interactif}
       {Empty line or EOF    $\Rightarrow$ interactive mode}\\
    \normalsize
    \item \fe{Ligne de \red{commentaire} : commence par \kwr{*}}
             {\red{Comment} line: starts with a \kwr{*}}
    \item \fe{Lignes vides autorisées}{Empty lines accepted}
  \end{itemize}
\end{frame}

\begin{frame}{\fe{Gibiane : règles de syntaxe}{Gibiane: syntax rules}}
  \begin{itemize}
    \item \fe{\red{Pas de priorité des opérations (lecture de gauche à droite)}}
             {\red{No precedence in operators (from left to rigth)}}\\
    \footnotesize
    \kw{1}\kwr{+}\kw{2}\kwr{*}\kw{3~~~= 9}\\
    \kw{1}\kwr{+}\kw{(2}\kwr{*}\kw{3) = 7}
    \normalsize
    \item \fe{Quelques \red{interdictions} :}{\red{Prohibitions}:}\\
    \footnotesize
    \fe{Pas de \red{tabulations} $\Rightarrow$ messages d'erreur incompréhensibles}
       {No \red{tab key} $\Rightarrow$ incomprehensible error message}\\
    \fe{Pas de \red{double quotes} \kw{"}}{No \red{double quote} \kw{"}}
    \normalsize
    \item \fe{Quelques (fortes !) \green{recommandations} :}{\green{Guidelines}:}\\
    \footnotesize
    \fe{Pas de \green{caractères spéciaux (é, ç, œ, \dots)}}{No special characters (é, ç, œ, \dots)}\\
    \fe{Utiliser une \green{indentation} (avec des espace)}
       {Use line indentation (with spaces)}\\
    \fe{Régler son éditeur de texte : coloration syntaxique, remplacement des tabulations par des espaces}
       {Adjust the text editor: syntax highlighting, switch tabulations by spaces}
    \normalsize
    \item \fe{Quelques \blue{erreurs} classiques :}{Common \blue{errors}:}\\
    \footnotesize
    \fe{Point virgule \kw{;} oublié en fin de ligne $\Rightarrow$ la lecture de l'instruction continue !}
       {Semicolon \kw{;} forgotten at the end of the statement $\Rightarrow$ the statement is not ended}\\
    \fe{Apostrophe \kw{'} oubliée à la fin d'une chaine $\Rightarrow$ la définition de la chaine continue !}
       {Single quote \kw{'} forgotten at the end of a string $\Rightarrow$ the string is not ended}
    \normalsize
  \end{itemize}
\end{frame}

\begin{frame}{\fe{Gibiane : les objets}{Gibiane: objects}}
  \begin{itemize}
    \item \fe{Définition}{Definition}\\
    \footnotesize
    \fe{Désigne toute \g{structure de données/résultats} munie d'un \red{type}\\
        (éventuellement d'un sous-type) et d'un \red{nom}}
       {Any data/result with a defined \red{type}\\
        (possibly a sub-type) and \red{name}}
    \normalsize
    \item \fe{Noms des objets}{Objects Names}\\
    \footnotesize
    \fe{Donné par l'utilisateur}{User Defined}\\
    \fe{\red{Limité à 24 caractères} parmi :}{\red{Limited to 24 characters} chosen in:} a...z A...Z 0...9 \_\\
    \fe{\red{Pièges classiques} :}{\red{Classic traps}:}\\
    \fe{~~~~plus de 24 caractères : les surnuméraires sont ignorés}{more than 24 characters: additional characters ignored}\\
    \fe{~~~~utilisation du tiret – $\Rightarrow$ interdit !}{dash sign – $\Rightarrow$ not allowed}\\
    \fe{~~~~caractères accentués é, è $\Rightarrow$ interdit !}{letters with accents: é, è $\Rightarrow$ not allowed}
    \normalsize
    \item \fe{Type des objets}{Objects types}\\
    \footnotesize
    \fe{Il existe plus de 40 types d'objets différents}{There are more than 40 types of objects}
    \normalsize
  \end{itemize}
\end{frame}

\begin{frame}{\fe{Gibiane : les objets}{Gibiane: objects}}
  \begin{itemize}
    \item \fe{Exemple}{Example}\\
    \lstinputlisting[language=gibiane, firstline=1, lastline=9]{dgibi/exemples.dgibi}
    \lstinputlisting[firstline=10, lastline=16]{dgibi/exemples.dgibi}
  \end{itemize}
\end{frame}

\begin{frame}{\fe{Gibiane : les opérateurs}{Gibiane: operators}}
  \begin{itemize}
    \item \fe{Définition}{Definition}\\
    \footnotesize
    \fe{Désigne tout \g{traitement} muni d'un \red{nom} (instruction Gibiane) qui
        construit un ou plusieurs \g{objets nouveaux} à partir d'un ou plusieurs
        objets existants}
       {Any \g{processing} with a \red{name} (Gibiane instruction) that creates
        \red{new object(s)} from pre-existing object(s)}
    \normalsize
    \item \fe{Noms des opérateurs}{Operators Names}\\
    \footnotesize
    \fe{Imposés à l'utilisateur}{Pre-defined}\\
    \fe{Ce sont des instructions Gibiane}{These are Gibiane instructions}\\
    \fe{Insensibles à la casse}{Case insensitive}\\
    \fe{Cast3M ne lit que les \red{4 premiers caractères} :}
       {Only the \red{4 first characters} are necessary and taken into account:}
    \kwr{DROITE} $\Leftrightarrow$ \kwr{DROI}\\
    \fe{Quelques exceptions : forme abrégée}{Excepted abbreviations}\\
    ~~~~\kwr{DROI} $\Leftrightarrow$ \kwr{D}\\
    ~~~~\kwr{CERC} $\Leftrightarrow$ \kwr{C}\\
    ~~~~\kw{poin2 =~}\kwr{POIN}\kw{ 28.~3.~;} $\Leftrightarrow$ \kw{poin2 = 28.~3.~;}\\
    ~~~~\kw{obj3~ =~}\kwr{MOT}\kw{~'Hello' ;} $\Leftrightarrow$ \kw{obj3~ = 'Hello' ;}
    \normalsize
  \end{itemize}
\end{frame}

\begin{frame}{\fe{Gibiane : les opérateurs}{Gibiane: operators}}
  \begin{itemize}
    \item \fe{Exemples d'appel à un opérateur (invocation)}
             {Examples of operator call}\\
    \footnotesize
    \fe{Cas courants : 1 objet à gauche du =}
       {Common cases: single object on the left of =}\\
    ~~~~\kw{obj1 = }\kwr{OPER}\kw{ obj2 ;}\\
    ~~~~\kw{obj3 = }\kwr{OPER}\kw{ obj4 obj5 ;}\\
    ~~~~\kw{obj6 = obj7 }\kwr{OPER}\kw{ obj8 obj9 ;}\\~\\
    \fe{Cas exceptionnels : plusieurs objets à gauche du =}
       {Unusual cases: multiple objects on the left of =}\\
    ~~~~\kw{obj1 obj2 obj3 = }\kwr{OPER}\kw{ obj4 obj5 ;}
    \normalsize
  \end{itemize}
\end{frame}

\begin{frame}{\fe{Gibiane : les opérateurs}{Gibiane: operators}}
  \begin{itemize}
    \item \fe{L'ordre des opérandes}
             {Arguments order}\\
    \footnotesize
    \fe{Est \green{indifférent} si les opérandes sont de \green{type différents}\\
        (sauf exception dans la documentation)\\
        Est \red{important} si plusieurs opérandes du \red{même type}}
       {\green{Do not matters} if arguments have \green{different types}\\
        (with a few exceptions pointed in the manual)\\
        \red{Matters} if \red{same type} arguments}
    \normalsize
    \item \fe{Surcharge d'un objet}
             {Overwriting an object}\\
    \footnotesize
    \fe{Toujours possible, l'ancien objet disparait}
       {Always possible, the overwritten object does not exists any longer}\\
       ~~~~\kw{A = 'Hello' ;} \fe{\kw{A} est du type MOT}{\kw{A} has type MOT}\\
       ~~~~\kw{B = 28 ;}\\
       ~~~~\kw{C = 3 ;}\\
       ~~~~\kw{A = B}\kwr{**}\kw{C ;}
           \fe{\kw{A} est du type ENTIER et vaut 21952}
              {\kw{A} has type ENTIER, its value is 21952}
    \normalsize
    \item \fe{Pièges classiques}{Classic traps}\\
    \footnotesize
    \fe{Nom d'objet = nom d'opérateur $\Rightarrow$ appel à l'opérateur impossible !\\
        sauf si on l'appelle entre quotes et en capitales}
       {Object name = operator name $\Rightarrow$ operator cannot be called!\\
        excepted if you call it with quotes in upper case}\\
    \kw{A = 'OPER' B C ;}\\
    \fe{Objet nommé \kw{C} ou \kw{D}}{Objet name \kw{C} ou \kw{D}}
    \normalsize
  \end{itemize}
\end{frame}

\begin{frame}{\fe{Gibiane : les directives}{Gibiane: directives}}
  \begin{itemize}
    \item \fe{Définition}{Definition}\\
    \footnotesize
    \fe{Commande sans symbole d'affectation =\\
        Ne crée pas de nouvel objet}
       {Statement without assignment operator =\\
       Does not create a new object}
    \normalsize
    \item \fe{Exemples}{Examples}\\
    \lstinputlisting[language=gibiane, firstline=18, lastline=21]{dgibi/exemples.dgibi}
  \end{itemize}
\end{frame}

\begin{frame}{\fe{Gibiane : les procédures}{Gibiane: procedures}}
  \begin{itemize}
    \item \fe{Définition}{Definition}\\
    \footnotesize
    \fe{Ensemble nommé de \g{commandes Gibiane} muni d'une liste d'opérandes d'entrée et de sortie\\
        Analogue à une subroutine Fortran ou à une fonction C}
       {Set of \g{Gibiane statements} having a name with input and output arguments\\
        Similar to Fortran subroutine or a C function}
    \normalsize
    \item \fe{Nom des procédures}{Procedures names}\\
    \footnotesize
    \fe{Comme un objet ordinaire (une procédure est un objet de type PROCEDUR)}
       {As other objects (a procedure is an object with PROCEDUR type)}
    \normalsize
    \item \fe{Déclaration}{Declaration}\\
    \lstinputlisting[language=gibiane, firstline=23, lastline=27]{dgibi/exemples.dgibi}
  \end{itemize}
\end{frame}

\begin{frame}{\fe{Gibiane : les procédures}{Gibiane: procedures}}
  \begin{itemize}
    \item \fe{Invocation}{Calling}\\
    \footnotesize
    \fe{Comme un opérateur ou une directive ordinaire}
       {As an operator or directive}\\
    \kw{a~b~c~=~}\kwr{MATHS}\kw{~21~11.0~;}
    \normalsize
    \item \fe{Il existe des procédures pré-cablées dans Cast3M}
             {Pre-existing procedures in Cast3M}\\
    \footnotesize
    \kwr{PASAPAS}  \fe{calculs non linéaires}{non-linear calculations}\\
    \kwr{FLAMBAGE} \fe{calculs de flambage}{buckling calculations}\\
    \kwr{DYNAMIC}  \fe{calculs dynamiques}{dynamics calculations}\\
    \kwr{THERMIC}  \fe{calculs thermiques}{thermal calculations}\\
    \kwr{G\_THETA} \fe{calcul d'intégrales J et FIC (rupture)}{J integrals and SIF computation (fracture)}\\
    \dots\\~\\
    \fe{Consultez la documentation :}{Read the doc:}\\
    \url{http://www-cast3m.cea.fr/index.php?page=notices}
    \normalsize
  \end{itemize}
\end{frame}

\begin{frame}{\fe{Gibiane : les procédures}{Gibiane: procedures}}
  \begin{itemize}
    \item \fe{Pièges classiques}{Classic traps}\\
    \footnotesize
    \fe{\kwr{FINP} \red{manquant}\\
        ~~~~$\Rightarrow$ arrêt de Cast3M + message d'erreur parfois difficile à interpréter}
       {\kwr{FINP} \red{missing}\\
        ~~~~$\Rightarrow$ Cast3M stops, error message that can be misunderstood}\\~\\
    \fe{\kwr{FINP} présent mais \kwr{;} manquant\\
        ~~~~$\Rightarrow$ arrêt de Cast3M + message d'erreur parfois difficile à interpréter}
       {\kwr{FINP} existinf but missing \kwr{;}\\
        ~~~~$\Rightarrow$ Cast3M stops, error message that can be misunderstood}\\~\\
    \fe{\red{Invocation} d'une procédure \red{avant} qu'elle ne soit \red{déclarée}\\
        ~~~~$\Rightarrow$ arrêt de Cast3M + message de l'opérateur =}
       {Procedure \red{called before} it is \red{declared}\\
        ~~~~$\Rightarrow$ Cast3M stops, error message in the = operator}\\
    \normalsize
  \end{itemize}
\end{frame}

\begin{frame}{\fe{Gibiane : quelques instructions utiles}{Gibiane: some useful statements}}
  \begin{itemize}
    \item \fe{Débogage}{Debugging}\\
    \footnotesize
    \fe{~}{\kwr{OPTI}\kwg{ 'LANG' 'ANGL'}\kw{ ;}}\\
    \kwr{INFO}\kw{ OPER ;}\\
    ~~~~$\Rightarrow$\fe{affiche la notice d'un opérateur/directive/procédure}
                        {print the manual page of a operator/directive/procedure}\\
    \kwr{OPTI}\kwg{ 'DONN'}\kw{ 5 ;}\\
    ~~~~$\Rightarrow$\fe{arrêt de la lecture du fichier .dgibi, passage en \green{mode interactif}}
                        {stop to run the file .dgibi switch to \green{interactive prompt}}\\
    \kwr{OPTI}\kwg{ 'DONN'}\kw{ 3 ;}\\
    ~~~~$\Rightarrow$\fe{reprise de la lecture du fichier .dgibi}
                        {return to run the file .dgibi}\\
    \kwr{LIST}\kw{ obj1 ;}\\
    ~~~~$\Rightarrow$\fe{affiche le contenu de l'objet}
                        {print object contents}\\
    \kwr{LIST}\kwg{ 'RESU'}\kw{ obj1 ;}\\
    ~~~~$\Rightarrow$\fe{affiche un \green{résumé} du contenu de l'objet}
                        {print a \green{summary} of the object}\\
    \kwr{OPTI}\kwg{ 'DEBU'}\kw{ 1 ;}\\
    ~~~~$\Rightarrow$\fe{accès aux variables locales des procédures}
                        {acces to procedure local variables}\\
    \kwr{TRAC}\kw{ obj1 ;}\\
    ~~~~$\Rightarrow$\fe{affiche l'objet}{plot the object}\\
    \kwr{MESS}\kwg{ 'Hello'}\kw{ ;}\\
    ~~~~$\Rightarrow$\fe{écrit un message}{print a message}\\
    \normalsize
  \end{itemize}
\end{frame}

\begin{frame}{Documentation}
  \begin{itemize}
    \item \fe{Notices des opérateurs/directives/procédures}{Manual pages of operators/directives/procedures}\\
    \begin{enumerate}
      \item \fe{Utiliser la \green{directive INFO} :}{Use the \green{INFO directive}:}
      \kwr{INFO}\kwg{ 'DROI'}\kw{ ;}\\~
      \item \fe{Consulter la \green{page html locale}}{Read the \green{local html page}}\\
      \fe{située dans le répertoire d'installation}{located in the installation directory}\\
      \fe{sur Linux :}{on Linux:}~~~~ \kw{/home/user/CAST3M\_2024/doc/index.html}\\
      \fe{sur Windows :}{on Windows:} \kw{C:\textbackslash{}Cast3M\textbackslash{}PCW\_24\textbackslash{}doc\textbackslash{}index.html}\\~
      \item \fe{Consulter le \green{site web}}{Read the \green{web site}}
      \url{http://www-cast3m.cea.fr/index.php?page=notices}\\
      \fe{\red{attention, il s'agit de la version du jour !}}{\red{dedicated to the up to date version!}}
    \end{enumerate}
    \item \fe{Manuels utilisateurs}{Users manual}\\
    \footnotesize
    \fe{sur le site web, à l'onglet "Documentation"}{On the web site, see the "Documentation" tab}
    \normalsize
  \end{itemize}
\end{frame}

\begin{frame}{\fe{Organisation d'un calcul élément-finis (4 grandes étapes)}
                 {Method to perform finite element calculations (4 main steps)}}
  \begin{enumerate}
    \footnotesize
    \item \fe{Choix de la géométrie et du maillage}
             {Geometry description and meshing}
    \begin{enumerate} \scriptsize
      \item \fe{Définition des points lignes, surfaces, volumes}
              {Description of points, lines, surfaces, volumes}
      \item \fe{Discrétisation}{Discretization}
    \end{enumerate}
    \item \fe{Définition du modèle mathématique}{Mathematical model definition}
    \begin{enumerate} \scriptsize
      \item \fe{Modèle (type d'analyse, formulation, comportement matériau, type d'élément)}
               {Model (type of analyze, formulation, material behavior, types of elements)}
      \item \fe{Propriétés matérielles (module d'Young, masse volumique, …)}
               {Material properties (Young's modulus, density, …)}
      \item \fe{Propriétés géométriques (épaisseur, moments quadratiques, …)}
               {Geometrical properties (thickness, quadratic moments of area, …)}
      \item \fe{Conditions aux limites/chargements}
               {Boundary conditions and loadings}
      \item \fe{Conditions initiales}
               {Initial values}
    \end{enumerate}
    \item \fe{Résolution du problème discrétisé}{Solving the discretized model}
    \begin{enumerate} \scriptsize
      \item \fe{Calcul des matrices de rigidité/masse élémentaires}
              {Elementary stiffness/mass matrix calculation}
      \item \fe{Assemblage des matrices}{Global matrix calculation}
      \item \fe{Application des conditions limites/chargements}
              {Introduction of the loadings and boundary conditions}
      \item \fe{Résolution du système d'équations}{Solving the system of equations}
    \end{enumerate}
    \item \fe{Analyse et post-traitement des résultats}{Analysis et post-processing}
      \begin{enumerate} \scriptsize
        \item \fe{Calcul de quantités locales (déplacement, contraintes, déformation, …)}
                 {Local quantities (strains, stresses, displacements, …)}
        \item \fe{Calcul de quantités globales (déformation maximale, charge limite, …)}
                 {Global quantities (maximal strains, strain energy, …)}
      \end{enumerate}
  \end{enumerate}
  \normalsize
\end{frame}

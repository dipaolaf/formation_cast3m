%%%%%%%%%%%%%%%%%%%%%%%%%%%%%%%%%%%%%%%%%%%%%%%%%%%%%%%%%%
\fe{\section{Langage Gibiane}}{\section{Gibiane Language}}
\label{gibiane}
%%%%%%%%%%%%%%%%%%%%%%%%%%%%%%%%%%%%%%%%%%%%%%%%%%%%%%%%%%

\begin{frame}{\fe{Le langage Gibiane}{The Gibiane language}}
  \begin{itemize}
    \item \fe{\red{Langage de programmation} destiné au calcul EF}
             {\red{Programming language} dedicated to FE calculation}\\
    \footnotesize
    \begin{textblock*}{5cm}(7cm,0.5cm)
      \includegraphics[width=3cm]{images/gibi_genie}
    \end{textblock*}
    \fe{Objets classiques (entiers, flottants, chaines, logiques, tables)\\
        Instructions conditionnelles\\
        Boucles itératives\\
        Sous structuration\\
        Récursivité}
       {Classical objects (integer, floating-point, string, logical , tables)\\
       Flow control statements\\
       Loops and iterations\\
       Subroutines\\
       Recursion}
    \normalsize
    \item \fe{Langage \red{interprété}}{\red{Interpreted} language}\\
    \footnotesize
    \fe{Le programme peut être exécuté dès que le script est modifié\\
        Le programme peut être exécuté en mode interactif}
       {You can run the program as soon as you make changes to the file\\
       You can run it in a interactive mode}
    \normalsize
    \item \fe{Langage \red{orienté objet}}{\red{Object-oriented} language}\\
    \footnotesize
    \fe{Tout est traité comme un objet\\
        Pas besoin de déclarer les variables ou de spécifier leur type}
       {Everything in the program is treated as an object\\
       No need to declare variables or to specify the type of a variable}
    \normalsize
    \item \fe{Mots clefs en français}{French key words}
    \item \fe{Programmation facile et rapide}{Easy to learn, easy to read}
  \end{itemize}
\end{frame}

\begin{frame}{\fe{Gibiane : règles de syntaxe}{Gibiane: syntax rules}}
  \begin{itemize}
    \item \fe{Ligne(s) de commande}{Statements lines}\\
    \footnotesize
    \fe{\red{500 caractères max} par instruction\\
        Une instruction peut être écrite sur plusieurs lignes\\
        Se \red{termine par un point virgule \kw{;}}\\
        Le \red{symbole d'affectation} est le signe égal \kwr{=}}
       {\red{500 characters max} per statement\\
        A statement can be written on several lines\\
        \red{End with a semicolon \kw{;}}\\
        The \red{assignment operator} is the equals sign \kwr{=}}
    \normalsize
    \item \fe{Insensibilité à la casse}{Case insensitive}\\
    \footnotesize
    \kw{TOTO = 3.14 ;}\\
    \kw{A~~~ = 2.~*~tOTo ;}
    \fe{la variable \kw{A} vaut bien 6.28}{variable \kw{A} has the value of 6.28}\\
    \normalsize
    \item \fe{Fin du fichier de données}{Program ends}\\
    \footnotesize
    \fe{commande \kwr{FIN ;} $\Rightarrow$ arrêt de Cast3M}
       {statement \kwr{FIN ;} $\Rightarrow$ Cast3M stop}\\
    \fe{ligne vide ou un EOF $\Rightarrow$ mode interactif}
       {empty line or EOF    $\Rightarrow$ interactive mode}\\
    \normalsize
    \item \fe{Ligne de \red{commentaire} : commence par \kwr{*}}
             {\red{Comment} line: starts with a \kwr{*}}
    \item \fe{Lignes vides autorisées}{Empty lines accepted}
  \end{itemize}
\end{frame}

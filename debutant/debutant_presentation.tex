%%%%%%%%%%%%%%%%%%%%%%
\section{Présentation}
%%%%%%%%%%%%%%%%%%%%%%
\begin{frame}{\fe{Cast3M, quid ?}{What is Cast3M?}}
  \fe{Logiciel de simulation utilisant la \g{méthode des éléments finis} en \g{mécanique/thermique} des \g{structures} et des \g{fluides}}
     {A simulation software using the \g{finite element method} in \g{thermal and mechanical} analysis of \g{structures} and \g{fluids}}\pause
  \begin{itemize}
    \item \fe{Résolution d'\g{équations aux dérivées partielles}}
             {\g{Partial differential equations} solver}\pause
    \item \fe{\g{Système complet} : solveur, pré/post-processeur, visualisation, import/export des données\dots}
             {\g{Complete software}: solver, pre-processing and post-processing, visualization, reading/writing data\dots}\\
            ~\\
            \begin{center}
            \includegraphics[height=0.25\textheight]{images/plaque.1} \:
            \includegraphics[height=0.25\textheight]{images/plaque.2} \:
            \includegraphics[height=0.25\textheight]{images/plaque.3}
            \end{center}\pause
    \item \fe{Basé sur un langage de commande : \g{Gibiane} (orienté objet)}
             {Based on a programming language: \g{Gibiane} (objet-oriented)}\\
  \end{itemize}
\end{frame}

\begin{frame}{\fe{Nombreux domaines d'application}{Wide range of applications}}
  \small{
  \begin{itemize}
    \item<1->\fe{Mécanique des structures}{Structural mechanics}\\
    \footnotesize{
      \fe{\red{Quasi-statique} (non linéarités matériau, géométrie, conditions limites)}
         {\red{Quasi-static} (non linear behavior, geometry, boundary conditions)}\\
      \onslide<1>{
        \begin{textblock*}{5cm}(2cm,0.3cm)
          \includegraphics[height=0.4\textheight]{images/polycristal}
        \end{textblock*}
        \begin{textblock*}{5cm}(5cm,1.2cm)
          \includegraphics[height=0.4\textheight]{images/ressort}
        \end{textblock*}
        \begin{textblock*}{5cm}(8cm,2.1cm)
          \includegraphics[height=0.4\textheight]{images/galatee}
        \end{textblock*}
        \begin{textblock*}{5cm}(9.5cm,5.6cm)
          \tiny{\emph{S. Durand}}
        \end{textblock*}}
      \onslide<2->\fe{\orange{Contact/frottement}, \green{Flambage}}
                     {\orange{Contact/friction}, \green{Buckling}}\\
      \onslide<2>{
        \begin{textblock*}{5cm}(1cm,0.3cm)
          \includegraphics[height=0.25\textheight]{images/sac_a_dos.15}
        \end{textblock*}
        \begin{textblock*}{5cm}(4.8cm,0.3cm)
          \includegraphics[height=0.25\textheight]{images/sac_a_dos.32}
        \end{textblock*}
        \begin{textblock*}{5cm}(8.5cm,0.3cm)
          \includegraphics[height=0.25\textheight]{images/sac_a_dos.41}
        \end{textblock*}}
      \onslide<3->\fe{\blue{Dynamique} (temporelle, modale, interaction fluide structure)}
                     {\blue{Dynamic} (temporal, modal, fluid structure interaction)}\\
      \onslide<3>{
        \begin{textblock*}{5cm}(1cm,0.3cm)
          \includegraphics[height=0.2\textheight]{images/cymbale_mode_1}
        \end{textblock*}
        \begin{textblock*}{5cm}(4.8cm,0.3cm)
          \includegraphics[height=0.2\textheight]{images/cymbale_mode_2}
        \end{textblock*}
        \begin{textblock*}{5cm}(8.5cm,0.3cm)
          \includegraphics[height=0.2\textheight]{images/cymbale_mode_4}
        \end{textblock*}}
      \onslide<4->\fe{\violet{Rupture} (XFEM, propagation dynamique, zones cohésives)}
                     {\violet{Fracture} (XFEM, dynamic propagation, cohesive zones models)}\\
      \onslide<4>{
        \begin{textblock*}{12cm}(1.5cm,0.3cm)
          \includegraphics[height=0.25\textheight]{images/rousselier_03}
          \includegraphics[height=0.25\textheight]{images/rousselier_04}
          \includegraphics[height=0.25\textheight]{images/rousselier_05}
          \includegraphics[height=0.25\textheight]{images/rousselier_06}
          \includegraphics[height=0.25\textheight]{images/rousselier_07}
          \includegraphics[height=0.25\textheight]{images/rousselier_08}
        \end{textblock*}
        \begin{textblock*}{5cm}(8cm,4.7cm)
          \tiny{\emph{S. Kebiri}}
        \end{textblock*}}
    }
    \item<5->\fe{Thermique}{Thermal analysis}\\
    \footnotesize{
      \fe{Conduction, convection, rayonnement, changement de phase}
         {Conduction, convection, radiation, phase transition}
%       té
    }
    \item<6->\fe{Mécanique des fluides}{Fluid mechanics}
    \item<6->\fe{Magnétostatique}{Magneto-statics}
    \item<6->\fe{Diffusion multi espèces (loi de Fick)}{Multi species diffusion (Fick’s law)}
    \item<6->\fe{Métallurgie}{Metallurgy}
%    clément
    \item<6->\fe{Couplage thermo-hygro-mécanique}{Thermo-hygro-mechanics coupling}
  \end{itemize}
  }
\end{frame}

\begin{frame}{\fe{Comment obtenir Cast3M ?}{How to get Cast3M}}
  \begin{itemize}
    \item \fe{Multi plateformes}{Cross platform}\\
    \small \blue{Windows}, \red{Linux}, \green{macOS} \normalsize
    \item \fe{Où télécharger Cast3M ?}{Where can I download Cast3M?}\\
    \small \url{http://www-cast3m.cea.fr/index.php?page=dlcastem} \normalsize
    \item \fe{Accès au code source}{Access to the source code}\\
    \small
    \fe{Développement communautaire}{Open collaboration}\\
    \fe{Compilateur / éditeur de liens fournis}{Compiler / Linker are provided}
    \normalsize
    \item \fe{Prix}{Price}\\
    \small
      \fe{\green{Gratuit} pour la recherche et l’enseignement\\
          \red{Payant} pour une utilisation commerciale}
         {Free license, for education and research use\\
          Paid license, for enterprise use}
    \normalsize
    \item \fe{Quelques utilisateurs/clients}{Some users/customers}\\
    \small
    \fe{Universités, écoles d’ingénieurs\\
        IRSN, EDF, SNCF, CNRS, Framatome, Air Liquide, CERN, \dots\\
        \scriptsize
        Outil de référence IRSN pour les analyse de sureté des installations nucléaires françaises\\
        Outil de référence Framatome pour l’analyse en mécanique de la rupture}
       {Universities, engineering schools\\
        IRSN, EDF, SNCF, CNRS, Framatome, Air Liquide, CERN, \dots\\
        \scriptsize
        Reference FEM tool for IRSN for safety analysis of French nuclear installations\\
        Reference tool for Framatome for fracture mechanics}
        \normalsize
    \end{itemize}
\end{frame}
